% !TEX root = sommaire.tex
\Chapter{Notations}

\section*{Graphes}

\begin{tabular}{p{2.5cm} p{11cm}}
  \hline
  $G = (V, E)$ & Graphe avec ensemble de nœuds $V$ et arêtes $E$\\
  $N$ & Nombre de nœuds dans le graphe\\
  $v_i$ & Nœud $i$ du graphe\\
  $\mathcal{N}(i)$ & Voisinage du nœud $i$\\
  $\mathbf{A}$ & Matrice d'adjacence\\
  $\tilde{\mathbf{A}}$ & Matrice d'adjacence avec self-loops ($\tilde{\mathbf{A}} = \mathbf{A} + \mathbf{I}$)\\
  $\mathbf{D}$ & Matrice de degré\\
  $\tilde{\mathbf{D}}$ & Matrice de degré de $\tilde{\mathbf{A}}$\\
  $\mathbf{L}$ & Matrice Laplacienne\\
  $d_i$ & Degré du nœud $i$\\
  $k$ & Nombre de plus proches voisins (K-NN)\\
  \hline
\end{tabular}

\section*{Features et représentations}

\begin{tabular}{p{2.5cm} p{11cm}}
  \hline
  $\mathbf{x}_i$ & Coordonnées 3D du nœud $i$, $\mathbf{x}_i \in \mathbb{R}^3$\\
  $\mathbf{f}_i$ & Vecteur de features du nœud $i$\\
  $\mathbf{h}_i^{(k)}$ & Représentation latente du nœud $i$ à la couche $k$\\
  $\mathbf{H}$ & Matrice de features de tous les nœuds\\
  $\mathbf{e}_{ij}$ & Features de l'arête entre nœuds $i$ et $j$\\
  $\mathbf{m}_{ij}$ & Message du nœud $j$ vers le nœud $i$\\
  $d$ & Dimension de l'espace (typiquement $d=3$)\\
  $D_h$ & Dimension de l'espace latent\\
  $D_f$ & Dimension du vecteur de features\\
  \hline
\end{tabular}

\section*{GNN et apprentissage}

\begin{tabular}{p{2.5cm} p{11cm}}
  \hline
  $\phi_e$ & Fonction de message (edge function)\\
  $\phi_h$ & Fonction de mise à jour des features (node update)\\
  $\phi_x$ & Fonction de mise à jour des coordonnées (EGNN)\\
  $\rho$ & Fonction de décodage (readout)\\
  $\text{AGG}$ & Fonction d'agrégation (sum, mean, max)\\
  $\sigma(\cdot)$ & Fonction d'activation non-linéaire (ReLU, ELU, etc.)\\
  $\mathbf{W}$ & Matrice de poids à apprendre\\
  $\alpha_{ij}$ & Coefficient d'attention entre nœuds $i$ et $j$\\
  $\mathcal{L}$ & Fonction de perte\\
  $\eta$ & Taux d'apprentissage (learning rate)\\
  \hline
\end{tabular}

\section*{Organoïdes et cellules}

\begin{tabular}{p{2.5cm} p{11cm}}
  \hline
  $\mathcal{O}$ & Ensemble des organoïdes\\
  $O_i$ & Organoïde $i$\\
  $C$ & Nombre de cellules dans un organoïde\\
  $c_i$ & Cellule $i$\\
  $V_i$ & Volume de la cellule $i$\\
  $\Psi_i$ & Sphéricité de la cellule $i$ ($\Psi_i \in [0,1]$)\\
  $S_i$ & Surface de la cellule $i$\\
  $A_i$ & Aire de la cellule de Voronoï $i$ (données synthétiques)\\
  $I_i^k$ & Intensité du canal fluorescent $k$ pour la cellule $i$\\
  \hline
\end{tabular}

\section*{Processus ponctuels}

\begin{tabular}{p{2.5cm} p{11cm}}
  \hline
  $\lambda$ & Intensité d'un processus de Poisson\\
  $\lambda(\mathbf{x})$ & Fonction d'intensité (cas inhomogène)\\
  $K(r)$ & Fonction K de Ripley au rayon $r$\\
  $F(r)$ & Fonction F (distance plus proche voisin)\\
  $G(r)$ & Fonction G (distance entre points)\\
  $r$ & Rayon ou distance euclidienne\\
  $\mathbb{S}^2$ & Sphère unitaire en dimension 3\\
  $R$ & Rayon de la sphère\\
  \hline
\end{tabular}

\section*{Statistiques et évaluation}

\begin{tabular}{p{2.5cm} p{11cm}}
  \hline
  $y$ & Label vrai (ground truth)\\
  $\hat{y}$ & Label prédit\\
  $C$ & Nombre de classes\\
  $\text{Acc}$ & Accuracy (taux de bonnes classifications)\\
  $\text{Prec}$ & Précision\\
  $\text{Rec}$ & Rappel (recall, sensibilité)\\
  $F_1$ & F1-score (moyenne harmonique précision/rappel)\\
  $\text{AUC}$ & Area Under Curve (aire sous la courbe ROC)\\
  $\text{ECE}$ & Expected Calibration Error (erreur de calibration)\\
  $\mu$ & Moyenne statistique\\
  $\sigma$ & Écart-type\\
  \hline
\end{tabular}

\section*{Transformations géométriques}

\begin{tabular}{p{2.5cm} p{11cm}}
  \hline
  $T$ & Transformation géométrique\\
  $E(3)$ & Groupe euclidien (translations, rotations, réflexions)\\
  $E(n)$ & Groupe euclidien en dimension $n$\\
  $SO(3)$ & Groupe des rotations 3D\\
  $O(3)$ & Groupe orthogonal 3D (rotations et réflexions)\\
  $\mathbf{R}$ & Matrice de rotation\\
  $\mathbf{t}$ & Vecteur de translation\\
  \hline
\end{tabular}

\section*{Ensembles et espaces}

\begin{tabular}{p{2.5cm} p{11cm}}
  \hline
  $\mathbb{R}$ & Ensemble des nombres réels\\
  $\mathbb{R}^n$ & Espace euclidien de dimension $n$\\
  $\mathbb{N}$ & Ensemble des entiers naturels\\
  $\mathcal{P}(\mathcal{X})$ & Ensemble des parties de $\mathcal{X}$ (permutation-invariant sets)\\
  $|\cdot|$ & Cardinalité d'un ensemble ou norme d'un vecteur\\
  $\|\cdot\|$ & Norme euclidienne (distance L2)\\
  \hline
\end{tabular}
