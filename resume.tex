% !TEX root = sommaire.tex

%%------------------------------------------------ Résumé français -
\clearpage
\thispagestyle{empty}
\vspace*{-2cm}

\begin{center}
{\large\bfseries Résumé}\par
\vspace{0.4cm}
\end{center}

Les organoïdes, ces mini-organes cultivés \textit{in vitro}, révolutionnent la recherche biomédicale en offrant des modèles tridimensionnels qui reproduisent la complexité des tissus humains. Cependant, leur analyse reste largement tributaire de méthodes manuelles, lentes et sujettes à des biais d'interprétation. Ces structures, composées de cellules organisées en réseaux d'interactions spatiales et fonctionnelles, nécessitent des outils capables de capturer non seulement leur morphologie, mais aussi les relations cellulaires qui déterminent leur mode de fonctionnement. C'est dans ce contexte que les réseaux de neurones sur graphes (Graph Neural Networks, GNN) émergent comme une solution particulièrement adaptée, permettant de modéliser les organoïdes non plus comme des images statiques, mais comme des systèmes relationnels où chaque cellule est un nœud connecté à ses voisines par des liens reflétant des interactions biologiques.

Cette thèse propose une approche innovante pour la modélisation et la classification automatisée des organoïdes à partir de graphes cellulaires, en exploitant pleinement le potentiel des GNN. Contrairement aux méthodes classiques basées sur des descriptions manuelles ou des réseaux de neurones convolutifs, qui analysent les images pixel par pixel, les GNN permettent d'intégrer des informations structurelles et contextuelles, en représentant chaque organoïde comme un réseau où les nœuds encodent des propriétés cellulaires (taille, forme, expression de marqueurs) et les arêtes capturent les relations spatiales. Cette représentation relationnelle ouvre la voie à une classification plus fine et plus interprétable, capable de distinguer des phénotypes subtils – comme des stades précoces de différenciation ou des altérations pathologiques – qui échappent aux approches traditionnelles.

Pour surmonter les défis posés par la rareté des données annotées et la variabilité intrinsèque des organoïdes, cette thèse développe une pipeline complète, depuis la construction de graphes cellulaires à partir d'images de microscopie jusqu'à l'apprentissage robuste de modèles de GNN. Une attention particulière est portée à la génération de données synthétiques via des modèles génératifs de graphes, afin d'enrichir les jeux d'entraînement et d'explorer des scénarios rares ou extrêmes. 

Les applications de cette approche sont multiples : criblage à grande échelle de composés pharmaceutiques, diagnostic précoce de maladies à partir d'organoïdes dérivés de patients, ou encore optimisation des protocoles de culture pour standardiser la production d'organoïdes. À plus long terme, cette thèse jette les bases d'une analyse globale combinant imagerie, graphes cellulaires et données omiques, ouvrant la voie à une compréhension plus profonde des mécanismes biologiques sous-jacents et à des avancées en médecine personnalisée.

\vspace{0.5cm}
\motscles{Machine Learning, Deep Learning, Graph Neural Networks, Organoids.}

%%------------------------------------------------ Abstract anglais -
\clearpage
\thispagestyle{empty}
\vspace*{-2cm}

\begin{center}
{\large\bfseries Abstract}\par
\vspace{0.4cm}
\end{center}

Organoids—miniaturized, three-dimensional \textit{in vitro} cultures that replicate the complexity of human tissues—are revolutionizing biomedical research. Yet their analysis remains heavily reliant on manual methods that are time-consuming, low-throughput, and prone to interpretative bias. These structures, composed of cells organized into spatial and functional interaction networks, demand analytical tools capable of capturing not only their morphology but also the cellular relationships that govern their behavior. In this context, Graph Neural Networks (GNNs) emerge as a particularly well-suited solution, enabling organoids to be modeled not as static images but as relational systems, where each cell is a node connected to its neighbors via edges representing biological interactions.

This thesis introduces an innovative framework for the automated modeling and classification of organoids using cellular graphs, fully leveraging the potential of GNNs. Unlike conventional approaches—based on manual descriptors or convolutional neural networks (CNNs), which analyze images pixel-by-pixel—GNNs integrate structural and contextual information by representing each organoid as a network. In this framework, nodes encode cellular properties (e.g., size, shape, marker expression) while edges capture spatial relationships. This relational representation enables finer and more interpretable classification, capable of distinguishing subtle phenotypes—such as early differentiation stages or pathological alterations—that elude traditional methods.

To address challenges posed by limited annotated data and the intrinsic variability of organoids, this work develops a comprehensive pipeline, from constructing cellular graphs from microscopy images to robust GNN training. Particular emphasis is placed on synthetic data generation via graph generative models to augment training sets and explore rare or extreme scenarios.

The applications of this approach are far-reaching: high-throughput drug screening, early disease diagnosis from patient-derived organoids, and optimization of culture protocols to standardize organoid production. In the long term, this thesis lays the groundwork for holistic multi-modal analysis—integrating imaging, cellular graphs, and omics data—to deepen our understanding of underlying biological mechanisms and advance precision medicine.

\vspace{0.5cm}
\keywords{Machine Learning, Deep Learning, Graph Neural Networks, Organoids.}
