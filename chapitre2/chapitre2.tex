% !TEX root = ../sommaire.tex

\chapter{État de l'art : Organoïdes, Images et Graphes}

\section{Organoïdes : biologie et applications}

\subsection{Définitions et types d'organoïdes}

Les organoïdes sont des structures tridimensionnelles auto-organisées cultivées \textit{in vitro} à partir de cellules souches ou de tissus primaires. Selon la définition de Lancaster et Knoblich~\cite{Lancaster2014}, un organoïde doit satisfaire plusieurs critères : contenir plusieurs types cellulaires de l'organe qu'il représente, présenter une organisation spatiale similaire à celle de l'organe natif, et récapituler au moins certaines fonctions de l'organe.

\subsubsection{Classification par origine cellulaire}

Les organoïdes peuvent être classés selon l'origine des cellules utilisées pour leur génération. Les organoïdes dérivés de cellules souches pluripotentes (PSC, pour ``pluripotent stem cells'') sont produits à partir de cellules souches embryonnaires (ESC) ou de cellules souches pluripotentes induites (iPSC). Leur mise en culture requiert des protocoles de différenciation dirigée, souvent complexes, afin de reproduire les étapes du développement embryonnaire. Cette approche permet d'obtenir des organoïdes de tissus habituellement inaccessibles comme le cerveau ou la rétine. Elle offre l'avantage d'une source illimitée de matériel et autorise des manipulations génétiques aisées, mais elle est contrebalancée par l'immaturité fonctionnelle fréquente des organoïdes ainsi produits, ainsi que par la durée des protocoles, qui s'étendent souvent sur plusieurs semaines.

Les organoïdes issus de cellules souches adultes (ASC, pour ``adult stem cells'') sont générés à partir de cellules souches résidentes dans des tissus adultes, comme les cryptes intestinales ou les glandes gastriques. Leur croissance se fait typiquement dans une matrice extracellulaire telle que le Matrigel, enrichie en facteurs de croissance adaptés au tissu d'origine. Ces organoïdes conservent généralement une identité tissulaire fidèle et présentent une maturation fonctionnelle supérieure à celle des PSC-derived. De plus, les protocoles de culture sont souvent plus simples et rapides. Cependant, cette méthode reste limitée par l'accès parfois difficile à certains tissus et par une capacité d'expansion qui varie selon le type cellulaire.

Enfin, on distingue une catégorie particulière d'organoïdes, les PDO (``patient-derived organoids''), issus directement de biopsies de patients. Ceux-ci constituent en réalité une sous-catégorie des organoïdes ASC-derived. Ils préservent les profils génétiques et épigénétiques propres au patient, ce qui les rend particulièrement précieux pour des applications en oncologie personnalisée (notamment les tumoroïdes). La constitution de biobanques d'organoïdes de patients permet ainsi de représenter la diversité des pathologies et d'ouvrir la voie à des études fonctionnelles individuelles.

\subsubsection{Classification par type d'organe}

Différents types d'organoïdes ont été développés avec succès :

\textbf{Organoïdes intestinaux :}
Premier système organoïde développé~\cite{Sato2009}, ils reproduisent l'architecture des villosités intestinales avec cryptes et cellules différenciées (entérocytes, cellules de Paneth, cellules entéroendocrines, cellules caliciformes). Utilisés pour étudier l'homéostasie intestinale, les maladies inflammatoires, les infections (SARS-CoV-2~\cite{Zhou2020COVID}), les pathologies génétiques (mucoviscidose~\cite{Dekkers2013}), et pour le criblage de drogues.

\textbf{Organoïdes cérébraux :}
Structures complexes mimant le développement cérébral précoce~\cite{Lancaster2013}, contenant différentes régions cérébrales (cortex, hippocampe, plexus choroïde). Applications en neurobiologie du développement, modélisation de maladies neurologiques (microcéphalie, autisme, schizophrénie), et étude de l'impact de pathogènes (virus Zika).

\textbf{Organoïdes hépatiques :}
Reproduisent l'architecture lobulaire du foie avec hépatocytes fonctionnels~\cite{Huch2015}. Applications en toxicologie (prédiction d'hépatotoxicité), métabolisme de drogues, modélisation d'hépatites virales et de maladies métaboliques.

\textbf{Organoïdes rénaux :}
Contiennent des structures néphroniques avec tubules et podocytes~\cite{Takasato2015}. Utilisés pour modéliser les maladies rénales génétiques et acquises, tester la néphrotoxicité de composés.

\textbf{Organoïdes pulmonaires :}
Modèles des voies aériennes et des alvéoles~\cite{Sachs2019}. Applications pour les infections respiratoires, la mucoviscidose, et le cancer du poumon.

\textbf{Autres organoïdes :}
Pancréas~\cite{Boj2015}, rétine~\cite{Lin2020}, estomac~\cite{Bartfeld2015}, prostate, glandes salivaires, sein~\cite{Sachs2018}, etc. La diversité croissante reflète l'universalité de l'approche.

Les organoïdes étudiés dans cette thèse sont des organoïdes de prostate.

\subsection{Mécanismes de formation et auto-organisation}

\subsubsection{Principes d'auto-organisation cellulaire}

La formation d'organoïdes repose sur les capacités intrinsèques d'auto-organisation cellulaire, gouvernées par plusieurs principes fondamentaux :

\textbf{Signalisation morphogénétique :}
Les cellules répondent à des gradients de molécules signal (Wnt, BMP, FGF, Notch) qui guident leur différenciation et leur positionnement spatial. En fournissant exogènement ces facteurs dans des combinaisons appropriées, on peut diriger le développement vers des destins cellulaires spécifiques.

\textbf{Interactions cellule-cellule :}
Les jonctions adhérentes (cadhérines), jonctions serrées, et gap junctions permettent aux cellules de communiquer, de s'organiser en épithéliums polarisés, et de coordonner leur comportement collectif.

\textbf{Interactions cellule-matrice :}
La matrice extracellulaire (ECM), fournie exogènement sous forme de Matrigel ou de matrices synthétiques, fournit un support structural et des signaux biochimiques (intégrines) régulant la forme, la migration et la différenciation cellulaires.

\textbf{Forces mécaniques :}
Les tensions cytosquelettiques, les pressions osmotiques, et les forces contractiles contribuent à façonner l'architecture tridimensionnelle. La formation de lumens résulte notamment de l'apoptose centrale et de la polarisation cellulaire.

\subsubsection{Étapes de développement d'un organoïde}

Le développement typique d'un organoïde intestinal illustre ces principes :
\begin{enumerate}
    \item \textbf{Agrégation initiale} (Jour 0-2) : Les cellules s'agrègent en sphéroïdes compacts dans le Matrigel
    \item \textbf{Polarisation} (Jour 2-4) : Les cellules développent une polarité apico-basale, avec migration des noyaux
    \item \textbf{Formation du lumen} (Jour 4-6) : Apoptose des cellules centrales créant une cavité interne
    \item \textbf{Bourgeonnement} (Jour 6-10) : Formation de bourgeons cryptiques par prolifération asymétrique
    \item \textbf{Différenciation} (Jour 10+) : Émergence de lignages différenciés (cellules absorbantes, sécrétoires)
    \item \textbf{Maturation} (Semaines) : Complexification de l'architecture, maturation fonctionnelle
\end{enumerate}

Cette séquence développementale, reminiscente de l'embryogenèse, se produit de manière largement autonome une fois les conditions initiales établies.

\subsection{Applications en recherche et en médecine}

\subsubsection{Recherche fondamentale}

\textbf{Développement et morphogenèse :}
Les organoïdes permettent d'étudier \textit{in vitro} les mécanismes de formation des organes, précédemment accessibles uniquement via embryologie. Des questions fondamentales sur la régulation génétique, l'auto-organisation, l'émergence de la complexité peuvent être abordées avec manipulations génétiques et imagerie en temps réel.

\textbf{Biologie des cellules souches :}
La niche des cellules souches, leur maintenance, leur différenciation, et leur réponse aux signaux peuvent être étudiées dans un contexte tissulaire 3D physiologique.

\textbf{Interactions hôte-pathogène :}
Les organoïdes fournissent des modèles d'infection plus réalistes que les monocultures 2D. L'infection par virus (rotavirus, norovirus, SARS-CoV-2), bactéries (Helicobacter, Salmonella), ou parasites peut être étudiée avec imagerie en temps réel et analyses fonctionnelles.

\subsubsection{Criblage pharmacologique et drug discovery}

\textbf{Découverte de médicaments :}
Les organoïdes offrent un système intermédiaire entre les cellules 2D (trop simplistes) et les modèles animaux (coûteux, lents, éthiquement problématiques) pour tester l'efficacité de composés. Des plateformes robotisées permettent le criblage de bibliothèques de milliers de molécules.

\textbf{Tests de toxicité :}
Les organoïdes hépatiques et rénaux sont utilisés pour prédire l'hépatotoxicité et la néphrotoxicité de composés en développement, réduisant les échecs tardifs en phases cliniques.

\textbf{Repositionnement de médicaments :}
Tester systématiquement des drogues approuvées sur de nouveaux modèles de maladies pour identifier de nouvelles indications thérapeutiques.

\subsubsection{Médecine personnalisée et applications cliniques}

\textbf{Prédiction de réponse thérapeutique :}
Les organoïdes tumoraux dérivés de patients peuvent être testés contre un panel de chimiothérapies, thérapies ciblées, ou immunothérapies pour prédire \textit{ex vivo} la sensibilité du patient et guider le choix thérapeutique. Plusieurs études pilotes ont démontré une concordance significative entre réponse des organoïdes et réponse clinique des patients.

\textbf{Diagnostic et pronostic :}
Au-delà du traitement, les organoïdes peuvent servir d'outils diagnostiques. La capacité d'expansion d'organoïdes à partir d'une biopsie peut être pronostique. Les profils moléculaires d'organoïdes peuvent compléter les analyses anatomopathologiques traditionnelles.

\textbf{Biobanques d'organoïdes :}
Des biobanques nationales et internationales d'organoïdes (Hubrecht Organoid Technology, Human Cancer Models Initiative) sont constituées pour capturer la diversité génétique et phénotypique des pathologies humaines~\cite{Drost2019}, servant de ressources partagées pour la communauté scientifique.

\subsubsection{Médecine régénérative : promesses futures}

L'application des organoïdes à la transplantation et à la réparation tissulaire, bien qu'encore largement expérimentale, connaît des avancées notables. Par exemple, des essais précliniques explorent l'utilisation d'organoïdes de rétine dans le but de restaurer la vision. D'autres travaux portent sur les organoïdes de foie, évalués comme support fonctionnel temporaire chez des patients en insuffisance hépatique, ainsi que sur les organoïdes de peau, qui pourraient offrir une solution innovante pour les greffes dans le cas de brûlures étendues. Néanmoins, plusieurs défis majeurs demeurent à surmonter afin de rendre ces approches cliniquement viables, notamment la vascularisation des organoïdes – essentielle pour les structures de diamètre supérieur à un millimètre afin d'assurer un apport sanguin adéquat –, leur innervation, ainsi que leur intégration fonctionnelle et immunologique avec l'organisme hôte.

\section{Analyse d'images biomédicales 3D}

\subsection{Modalités d'imagerie pour organoïdes}

\subsubsection{Microscopie confocale}

La microscopie confocale à balayage laser (CLSM) constitue la méthode de référence pour l’imagerie des organoïdes~\cite{Litjens2017}. Son principe repose sur l’excitation point par point de la fluorescence par un faisceau laser focalisé. Un trou d’épingle (\textit{pinhole}) placé devant le détecteur permet de rejeter la lumière provenant des plans hors foyer, ce qui aboutit à une image optiquement sectionnée. En balayant le faisceau à travers l’échantillon et en déplaçant le plan focal selon l’axe Z, il est ainsi possible de collecter des séries d’images (stacks) permettant de reconstruire le volume 3D de l’échantillon.

La CLSM présente plusieurs atouts majeurs : elle offre une résolution latérale élevée (200 à 300 nm) et une résolution axiale de l’ordre de 500 à 800 nm, un rejet efficace de la lumière hors-foyer, ainsi que la possibilité de réaliser des acquisitions multicanales sur plusieurs fluorophores (jusqu’à 4 à 6 simultanément). De plus, cette technique est largement disponible dans les laboratoires de biologie.

En revanche, la microscopie confocale possède plusieurs limitations. L’acquisition d’un volume 3D est relativement lente et nécessite souvent plusieurs minutes par organoïde. La répétition des scans expose les échantillons à un risque important de photoblanchiment, et la phototoxicité reste un obstacle pour l’imagerie prolongée sur échantillon vivant. Enfin, la profondeur de pénétration de l’excitation est restreinte, particulièrement dans les tissus denses, et dépasse rarement 100 à 150 μm.

\subsubsection{Microscopie light-sheet (LSFM)}

La microscopie à feuillet de lumière (LSFM, pour Light-Sheet Fluorescence Microscopy) consiste à illuminer l’échantillon par le côté à l’aide d’un fin plan lumineux, tandis que la détection de la fluorescence s’effectue dans un axe perpendiculaire à celui de l’illumination~\cite{Huisken2004}. Cette disposition permet d’acquérir rapidement des images volumétriques tout en minimisant l’exposition globale de l’échantillon à la lumière. Grâce à ce principe, la LSFM atteint des vitesses d’acquisition très élevées, pouvant capturer plusieurs images par seconde, tout en limitant considérablement le photoblanchiment et la phototoxicité. Après une étape de clarification optique, elle permet aussi d’imager en profondeur des échantillons épais tels que les organoïdes, ce qui la rend particulièrement adaptée pour des expériences de type time-lapse sur de longues durées ou pour la visualisation de volumes importants, de l’ordre de plusieurs millimètres cubes.

Cependant, la mise en œuvre de la microscopie à feuillet de lumière requiert des équipements spécialisés qui restent encore relativement peu répandus dans les laboratoires de biologie. De plus, l’imagerie de tissus denses nécessite souvent une étape préalable de clarification optique, afin de garantir une pénétration adéquate de la lumière. Enfin, si la LSFM permet de couvrir de larges volumes, sa résolution demeure légèrement inférieure à celle offerte par la microscopie confocale classique.

\subsubsection{Microscopie multiphoton}

La microscopie multiphoton repose sur l’utilisation d’impulsions laser infrarouges de forte intensité permettant d’exciter les fluorophores grâce à l’absorption simultanée de deux photons. Cette technique présente plusieurs atouts majeurs : la longueur d’onde infrarouge utilisée permet une pénétration plus profonde dans les tissus, jusqu’à environ 1 mm, ce qui est particulièrement avantageux pour l’imagerie d’échantillons épais. Par ailleurs, l’excitation étant confinée au seul point focal, le photoblanchiment des fluorophores est considérablement réduit dans les zones hors focus, limitant ainsi les effets secondaires indésirables sur l’échantillon. La microscopie multiphoton rend également possible l’imagerie \textit{in vivo}, une caractéristique précieuse pour l’étude d’échantillons vivants.

Cependant, cette modalité comporte aussi des limitations importantes. Tout d’abord, l’équipement nécessaire est onéreux, les lasers femtoseconde Ti:Sapphire étant particulièrement coûteux à acquérir et à entretenir. L’acquisition des images s’avère également plus lente comparée à une microscopie confocale classique, ce qui peut représenter un frein pour des expériences à haut débit. Enfin, il est nécessaire d’utiliser des fluorophores spécifiquement adaptés à l’excitation multiphoton, ce qui peut restreindre le choix des marqueurs disponibles.

\subsection{Défis spécifiques de l'imagerie d'organoïdes}

\subsubsection{Résolution, bruit et artefacts}

\textbf{Diffusion de la lumière :}
Dans les tissus biologiques, la diffusion (scattering) de la lumière dégrade la résolution et le contraste avec la profondeur. Les structures périphériques sont mieux résolues que le cœur de l'organoïde.

\textbf{Hétérogénéité d'illumination :}
L'atténuation de la lumière crée des gradients d'intensité non-uniformes, compliquant la segmentation automatique. Les cellules profondes apparaissent plus sombres que les cellules superficielles.

\textbf{Aberrations optiques :}
Les différences d'indice de réfraction entre milieu de montage, Matrigel et tissu créent des aberrations sphériques et chromatiques dégradant la qualité d'image.

\textbf{Bruit photonique :}
À faibles niveaux de lumière (imagerie live pour réduire la phototoxicité), le bruit de Poisson des photons devient significatif, dégradant le rapport signal/bruit.

\subsubsection{Variabilité inter-acquisitions}

Les conditions d'imagerie présentent une grande variabilité d'une expérience à l'autre. Parmi les sources majeures de variation, on compte la puissance du laser, le gain appliqué au détecteur ou encore le temps d’exposition, qui peuvent fluctuer selon la configuration de l’appareillage et les réglages opérateurs. Les performances optiques elles-mêmes évoluent dans le temps avec l’alignement régulier requis des composants et le vieillissement des lasers, pouvant affecter la qualité des acquisitions. L’orientation de l’échantillon représente aussi un facteur non négligeable, d’autant plus que les organoïdes non fixés peuvent se déplacer lors de la manipulation ou au cours de l’imagerie. Enfin, les protocoles de marquage présentent une efficacité variable en fonction de la pénétration des anticorps, et le phénomène de photoblanchiment n’affecte pas toujours l’ensemble du volume de manière homogène. L’ensemble de ces variations complexifie le développement de méthodes d’analyse robustes et universellement applicables, à moins de recourir à des stratégies de normalisation adaptées et performantes.

\subsection{Contraintes computationnelles}

\subsubsection{Volumes de données}

Un organoïde imagé en haute résolution (objectif 40×, voxel 0.3×0.3×1 μm) génère typiquement 2048×2048x200 voxels, soit ~2 Go par organoïde.

Par exemple, pour une étude de criblage pharmacologique testant 100 composés × 10 concentrations × 5 réplicats × 3 temps = 15,000 organoïdes, le volume total atteint 30-60 To.

\subsubsection{Défis de traitement}

Le traitement de ces volumes pose plusieurs défis :

\textbf{Mémoire} : Charger une image complète en mémoire peut dépasser la RAM disponible (64-128 Go typiques)

\textbf{Temps de calcul} : L'analyse naïve voxel-par-voxel est prohibitivement lente

\textbf{Parallélisation} : Nécessité de stratégies de traitement par blocs (tiling) ou de streaming

\textbf{Stockage} : Coûts de stockage et de backup importants, nécessité de compression

Ces contraintes motivent le développement de représentations compactes et efficaces, comme les graphes cellulaires proposés dans cette thèse.

\section{Méthodes d'analyse existantes}

\subsection{Analyse manuelle : référence mais non scalable}

\subsubsection{Protocoles d'analyse experte}

L'analyse manuelle se déroule généralement selon plusieurs étapes complémentaires. L'observateur commence par explorer l'organoïde dans sa globalité à l'aide de dispositifs de visualisation 3D, tels que les rendus volumiques ou les projections en intensité maximale, afin d'appréhender l'architecture générale de l'échantillon. Il procède ensuite à une évaluation qualitative de la morphologie globale, repérant d'éventuelles anomalies ou caractéristiques saillantes. L'étude se poursuit par une inspection attentive de différentes sections 2D extraites à diverses profondeurs, ce qui permet de mieux caractériser l'organisation interne. La quantification des marqueurs d'intérêt, comme le comptage des cellules positives à un signal fluorescent donné, s'effectue de façon semi-manuelle via des logiciels spécialisés. Enfin, l'organoïde est classé d'après des critères prédéfinis ou bien positionné sur une échelle ordinale, selon le protocole établi pour l'étude.

\subsubsection{Coûts et contraintes pratiques}

\textbf{Coûts temporels :}
Pour une analyse détaillée comptant 15-30 min par organoïde, cela représente 250-500 heures, soit 1-2 mois de travail temps plein.

\textbf{Coûts financiers :}
À 50€/heure de travail d'un expert, l'analyse de 1000 organoïdes coûte 12,500-25,000€, motivant l'automatisation.

\subsection{Segmentation cellulaire : état de l'art}

La segmentation automatique des cellules constitue une étape critique précédant toute analyse quantitative. Plusieurs approches ont été développées.

\subsubsection{Méthodes classiques de segmentation}

\textbf{Seuillage et détection de blobs :}
Les approches les plus simples appliquent un seuil global ou adaptatif sur le canal nucléaire (DAPI), puis détectent les régions connectées. Limitations : fusion de noyaux proches, sensibilité au choix de seuil.

\textbf{Watershed :}
L'algorithme de watershed traite l'image comme une surface topographique où les intensités représentent l'altitude. Les minima locaux sont inondés progressivement jusqu'à ce que les bassins versants se rencontrent, définissant les frontières de segmentation. Problème majeur : sur-segmentation nécessitant un post-traitement (détection de seeds, marker-controlled watershed).

\textbf{Active contours et level sets :}
Des contours évoluent sous l'influence de forces internes (régularité) et externes (gradients d'intensité) pour épouser les frontières cellulaires. Méthodes élégantes mais lentes en 3D et sensibles à l'initialisation.

\subsubsection{Approches par deep learning}

\textbf{U-Net et variantes :}
L'architecture U-Net~\cite{Ronneberger2015}, introduite pour segmentation biomédicale 2D, a été étendue en 3D~\cite{Cicek2016}. L'architecture encoder-decoder avec skip connections permet la segmentation sémantique dense. Cependant, elle nécessite des annotations pixel-level coûteuses, ainsi qu'une empreinte mémoire importante en 3D.

\textbf{Mask R-CNN et détection d'instances :}
Détection et segmentation simultanées d'instances d'objets. Cette approche est applicable aux noyaux cellulaires mais présente des difficultés pour gérer les chevauchements en 3D.

\textbf{StarDist :}
Une approche innovante représentant chaque cellule par son centroïde et un champ de distances radiales (star-convex polygons)~\cite{Schmidt2018}, très performante pour les noyaux de forme convexe. Une implémentation 2D et 3D est disponible. Cependant, elle ne gère pas bien les formes non-convexes.

\textbf{Cellpose :}
État de l'art actuel~\cite{Stringer2021}, basé sur la prédiction de champs de gradients où chaque pixel/voxel "pointe" vers le centre de sa cellule. Le suivi de ces gradients permet de regrouper les pixels en cellules. Cette méthode présente plusieurs avantages majeurs qui en font un outil de référence. Elle se montre robuste aux variations de taille cellulaire et gère efficacement les morphologies irrégulières ainsi que les cellules denses. La version 3D est particulièrement efficace pour le traitement de volumes complets. De plus, Cellpose propose des modèles pré-entraînés généralistes offrant d'excellentes performances, avec la possibilité de réaliser un fine-tuning sur des données spécifiques pour améliorer encore les résultats sur des types cellulaires particuliers.

Cellpose représente actuellement le meilleur compromis précision/généralisation pour la segmentation de noyaux et de cellules en 3D.

\subsubsection{Bilan des méthodes de segmentation existantes}

L'état de l'art montre un trade-off fondamental entre précision et vitesse. Pour une précision maximale, Cellpose offre d'excellentes performances mais nécessite environ 30 secondes par coupe, ce qui le rend lent pour le criblage à haut débit. StarDist offre un compromis intéressant avec une précision moindre mais un temps de calcul de seulement 5 secondes par coupe. À l'opposé du spectre, les méthodes privilégiant la rapidité incluent le watershed, qui est rapide mais imprécis, et le seuillage simple, très rapide mais inadapté aux configurations où les cellules sont collées les unes aux autres.

Pour l'analyse de milliers d'organoïdes nécessaire à notre étude, Cellpose standard (30 sec/coupe × 200 coupes/organoïde × 1500 organoïdes = 2500 heures ≈ 104 jours) est prohibitivement lent.

Cette limitation motive le développement de méthodes de segmentation optimisées, présentées au Chapitre 4 comme contributions méthodologiques de la thèse.

\subsubsection{Évaluation et benchmarking}

Les méthodes de segmentation sont typiquement évaluées via :

\textbf{Métriques pixel-level} : Dice coefficient, Intersection over Union (IoU)

\textbf{Métriques objet-level} : Précision, rappel, F1-score sur détection d'instances

\textbf{Average Precision (AP)} : Métrique standard des défis de segmentation (issu de détection d'objets)

\textbf{Segmentation Covering} : Mesure de recouvrement entre segmentation prédite et vérité terrain

Les compétitions internationales (Cell Tracking Challenge~\cite{Ulman2017}, Data Science Bowl~\cite{Caicedo2019}) ont poussé l'état de l'art avec des datasets de référence annotés.

\subsection{Approches par vision par ordinateur classique}

\subsubsection{Descripteurs de texture}

Les organoïdes se distinguent par des textures spécifiques, telles que des profils d’intensité et des arrangements spatiaux particuliers, qui peuvent être quantifiés à l’aide de différents types de descripteurs.

Parmi les descripteurs de texture classiques en vision par ordinateur, on retrouve notamment les matrices de co-occurrence de Haralick, les Local Binary Patterns (LBP) et les filtres de Gabor~\cite{Haralick1973, Ojala2002, Fogel1989}. Les matrices de Haralick consistent à analyser la fréquence des paires de niveaux de gris à une distance et une orientation données dans l’image, ce qui permet d’extraire des attributs tels que le contraste, la corrélation, l’énergie, l’homogénéité ou l’entropie. Bien qu’une extension tridimensionnelle soit possible et parfois appliquée aux volumes d’organoïdes, elle reste coûteuse en temps de calcul, et ces descripteurs tendent à perdre l’information spatiale globale tout en restant assez génériques.

Les Local Binary Patterns offrent une approche complémentaire en codant, pour chaque pixel, les relations d’intensité entre celui-ci et ses voisins immédiats. Cette description locale, étendue en 3D, permet de capturer les micro-textures fréquentes dans certaines images biologiques. Les LBP sont appréciés pour leur invariance aux variations d’illumination, mais demeurent sensibles au bruit.

Les filtres de Gabor, enfin, sont utilisés pour détecter des structures directionnelles et des motifs texturés grâce à des convolutions avec des noyaux paramétrés en fréquence et en orientation. Appliqués en 2D ou en 3D, ils sont particulièrement adaptés à la caractérisation de textures périodiques et de patrons directionnels, au prix toutefois d’une charge computationnelle significative lorsque l’on considère des banques de filtres tridimensionnelles~\cite{Fogel1989}.

\subsubsection{Descripteurs géométriques et morphologiques}

\textbf{Moments géométriques :}
Moments d'ordre 0 (volume), 1 (centroïde), 2 (matrice d'inertie, axes principaux), 3+ (asymétrie, kurtosis). Les moments de Hu invariants par transformation peuvent caractériser la forme.

\textbf{Descripteurs de forme :}
\begin{itemize}
    \item Sphéricité : $\Psi = \frac{\pi^{1/3}(6V)^{2/3}}{S}$ où $V$ est le volume et $S$ la surface
    \item Excentricité : Rapport des axes principaux
    \item Compacité, convexité, solidité
    \item Descripteurs de Fourier de la surface
\end{itemize}

\textbf{Analyse de distribution spatiale :}
Gradients radiaux d'intensité (du centre vers périphérie), profils d'intensité le long d'axes, moments spatiaux d'ordre supérieur.

\subsubsection{Machine learning classique sur descripteurs}

Une fois les features extraites, des algorithmes de ML classique sont appliqués :

\textbf{Random Forest :}
Ensembles d'arbres de décision, robustes, gèrent bien les features hétérogènes et les non-linéarités. Fournissent des importances de features pour interprétabilité.

\textbf{Support Vector Machines (SVM) :}
Avec noyaux non-linéaires (RBF), performants pour classification avec données limitées. Sensibles au scaling des features.

\textbf{Gradient Boosting (XGBoost, LightGBM) :}
État de l'art pour données tabulaires, souvent supérieurs aux Random Forest, mais ils nécessitent le tuning d'hyperparamètres.

\textbf{Limitations principales :}
Ces approches souffrent cependant de plusieurs limitations importantes. Le plafond de performance est limité par l'expressivité des features extraites, et la conception de ces descripteurs nécessite une expertise domaine approfondie pour le feature engineering. De plus, ces méthodes entraînent une perte d'information relationnelle, car les relations spatiales entre les cellules ne sont pas capturées par des statistiques globales. Enfin, elles offrent peu de généralisation à d'autres types d'organoïdes, chaque type nécessitant souvent un ensemble de features spécifiques.

\subsection{Approches récentes spécifiques aux organoïdes}

Plusieurs travaux récents ont abordé spécifiquement l'analyse automatisée d'organoïdes par apprentissage profond.

\subsubsection{Outils de deep learning pour organoïdes}

Park et al.~\cite{Park2023} ont développé un outil de traitement d'images basé sur le deep learning pour une analyse améliorée d'organoïdes, démontrant l'utilité de l'apprentissage profond pour la quantification automatisée. De même, Haja et al.~\cite{Haja2023} ont proposé une approche de deep learning pour localiser et quantifier automatiquement les organoïdes dans des images, adressant le problème de passage à l'échelle.

Pour les organoïdes rénaux spécifiquement, Wilson et al.~\cite{Wilson2022} ont développé DevKidCC, un outil permettant une classification robuste et des comparaisons directes entre différents datasets d'organoïdes de rein, illustrant l'importance de la standardisation pour la reproductibilité.

Au-delà des approches purement basées sur l'apprentissage, Laussu et al.~\cite{Laussu2024} ont proposé un modèle d'éléments finis centré sur les cellules pour l'analyse de la forme d'organoïdes intestinaux, reliant architecture tissulaire et mécanique, démontrant la complémentarité entre approches physiques et computationnelles.

Ces travaux pionniers, bien que prometteurs, présentent des limitations : ils sont souvent spécifiques à un type d'organoïde, n'exploitent pas pleinement la structure relationnelle 3D, ou nécessitent des annotations manuelles importantes. Notre approche vise à adresser ces limitations via des représentations par graphes et du transfer learning.

\subsection{Deep learning pour les images biomédicales}

\subsubsection{CNN 2D : approches par slices}

\textbf{Max/Mean intensity projections :}
Une approche courante consiste à projeter le volume 3D en une image 2D en calculant, par exemple, la valeur maximale (maximum intensity projection) ou la moyenne (mean intensity projection) le long de l’axe Z, avant d’appliquer un réseau de neurones convolutifs 2D classique, comme ResNet ou EfficientNet. Cette méthode présente l’avantage d’être peu coûteuse en ressources de calcul, mais elle s’accompagne d’une perte importante d’information tridimensionnelle, car la structure 3D complexe n’est plus accessible à l’analyse.

\textbf{Multi-slice analysis :}
Une approche dite "multi-slice" consiste à extraire plusieurs coupes bidimensionnelles à différentes profondeurs du volume, puis à appliquer un classifieur à chacune de ces images individuelles. Les résultats obtenus pour chaque coupe sont ensuite agrégés, typiquement par un vote majoritaire ou une moyenne des scores de prédiction, afin d’obtenir une décision globale sur le volume analysé. Cette méthode permet de récupérer plus d’informations que les simples projections 2D puisqu’elle exploite la diversité des plans à travers l’organoïde, mais elle ne prend pas réellement en compte la continuité spatiale et les relations entre coupes, la cohérence tridimensionnelle n’étant donc pas modélisée.

\textbf{2.5D approaches :}
Une approche dite « 2.5D » consiste à empiler trois coupes adjacentes d’un volume pour former une image pseudo-RGB, qui peut alors être traitée par un réseau de neurones convolutifs 2D standard. Cette méthode permet de capturer une partie de l’information tridimensionnelle tout en conservant l’efficacité computationnelle des architectures 2D, constituant ainsi un compromis entre richesse spatiale et coût de calcul.

\subsubsection{CNN 3D : extension naturelle mais coûteuse}

Les CNN 3D étendent les opérations de convolution et pooling à trois dimensions.

\textbf{Architectures classiques 3D :}
\begin{itemize}
    \item \textbf{3D U-Net} : Segmentation sémantique dense en 3D, encoder-decoder avec skip connections
    \item \textbf{V-Net} : Variante avec residual connections et convolutions 3D
    \item \textbf{ResNet 3D, DenseNet 3D} : Adaptations des architectures 2D célèbres
\end{itemize}

\textbf{Contraintes pratiques :}
L’utilisation de réseaux de neurones convolutifs 3D se heurte à des contraintes mémoire importantes sur les GPU standards (16 à 32 Go), ce qui impose plusieurs ajustements pour pouvoir entraîner ces modèles. Il est généralement nécessaire de réduire fortement la taille des volumes en appliquant un downsampling drastique (par exemple, à 64×64×64 ou 128×128×128 voxels au maximum). Par ailleurs, il faut souvent traiter les données sous forme de patchs ou de sous-volumes extraits du volume initial, puis recombiner les résultats. La profondeur du réseau lui-même doit être limitée afin de réduire l’empreinte mémoire, et la taille des lots (batch size) pendant l’entraînement doit fréquemment être extrêmement réduite, parfois à seulement un ou deux échantillons par itération.

Le downsampling détruit les détails fins (cellules individuelles non résolues), limitant la capacité à capturer l'information biologique subtile.

\textbf{Résultats et limitations :}
Bien que les CNN 3D puissent obtenir de bonnes performances sur certaines tâches de classification d'organoïdes, ils présentent plusieurs limitations importantes. Ils nécessitent de larges datasets annotés comptant plusieurs milliers d'exemples pour atteindre de bonnes performances. De plus, ils sont sensibles aux variations d'acquisition (domain shift), ce qui limite leur capacité de généralisation entre différents protocoles expérimentaux. Ces modèles manquent également d'interprétabilité, fonctionnant comme des boîtes noires qui ne permettent pas d'identifier les caractéristiques biologiques discriminantes. Enfin, ils requièrent une augmentation extensive des données pour tenir compte des invariances géométriques, notamment les rotations et les symétries en 3D.

\subsection{Approches basées graphes en histopathologie}

\subsubsection{Contexte : graphes cellulaires en pathologie numérique}

Des travaux pionniers ont exploré l'utilisation de graphes pour l'analyse de tissus en histopathologie 2D~\cite{Zhou2019,Jaume2021,Pati2022}, anticipant notre approche.

\textbf{Graphes de cellules pour la classification de cancers :}
Les cellules dans une coupe histologique 2D sont représentées comme nœuds d'un graphe, connectées selon la proximité spatiale. Les features incluent morphologie nucléaire, texture, intensité de coloration. Des GNNs sont entraînés pour prédire le grade tumoral, le sous-type histologique, ou le pronostic.

\textbf{Graphes de tissus multi-échelles :}
Représentations hiérarchiques où les nœuds peuvent être des cellules individuelles, des régions tissulaires, ou des glandes entières. Capture d'informations multi-échelles pertinentes pour le diagnostic.

\textbf{Transcriptomique spatiale :}
Avec les technologies de transcriptomique spatiale, chaque "spot" (région de ~50 μm) est un nœud avec comme features le profil d'expression génique. Les GNNs intègrent l'information spatiale pour identifier des niches cellulaires, prédire des états cellulaires.

\subsubsection{Résultats prometteurs}

Ces approches ont montré des performances supérieures ou, du moins, comparables à celles des CNN classiques sur certaines tâches. Elles offrent également une meilleure interprétabilité, puisqu’il est possible d’identifier les cellules ou les régions tissulaires qui contribuent le plus à la décision. De plus, les méthodes basées sur les graphes se révèlent robustes face aux variations de taille d’image et permettent de capturer des motifs topologiques comme le regroupement cellulaire ou l’arrangement spatial, difficiles à appréhender avec des approches purement convolutionnelles.

\subsubsection{Limitations et extension aux organoïdes 3D}

Cependant, ces travaux présentent des limites importantes :

\textbf{Limitation 2D} : Les coupes histologiques sont intrinsèquement 2D, perdant l'information 3D cruciale des organoïdes

\textbf{Tissus plans} : Les approches sont conçues pour des architectures planaires, pas pour des structures sphéroïdales 3D

\textbf{Features simples} : Les graphes utilisés n'exploitent généralement pas les coordonnées spatiales 3D explicitement

\textbf{Pas d'équivariance géométrique} : Les architectures ne respectent pas les symétries 3D naturelles

L'extension de ces approches aux organoïdes 3D avec graphes géométriques équivariants constitue une contribution originale de notre travail.

\section{Positionnement de la thèse}

\subsection{Lacunes identifiées dans la littérature}

Notre revue de la littérature révèle plusieurs lacunes majeures :

\subsubsection{Absence de méthodes automatisées spécifiques aux organoïdes 3D}

À ce jour, il n’existe, à notre connaissance, aucune méthode publiée qui décrive un cadre complet et automatisé spécifiquement adapté à l’analyse d’organoïdes en trois dimensions. Les solutions actuellement disponibles présentent plusieurs limites : certaines, comme ImageJ ou CellProfiler, sont des outils généralistes qui requièrent une expertise technique importante de la part de l'utilisateur ; d’autres se concentrent sur des types de données différents, notamment l’histopathologie ou les cultures en deux dimensions ; certains pipelines proposent uniquement des modules isolés, tels que la segmentation sans analyse subséquente intégrée ; enfin, la plupart de ces outils n’ont pas fait l’objet de validations rigoureuses sur des jeux de données propres aux organoïdes 3D.

\subsubsection{Sous-exploitation de la structure relationnelle}

Les méthodes existantes traitent majoritairement les organoïdes comme des images, sans modéliser explicitement le réseau d'interactions cellulaires qui gouverne leur comportement. Les CNN opèrent sur des voxels, les descripteurs calculent des statistiques globales, mais la topologie du graphe cellulaire n'est pas exploitée.

\subsubsection{Manque de solutions au problème de données limitées}

La rareté de données annotées est un verrou reconnu mais peu adressé. Aucune approche de génération de données synthétiques spécifique aux organoïdes n'a été proposée. Les stratégies de transfer learning ou de few-shot learning pour ce domaine sont inexistantes.

\subsubsection{Absence de prise en compte des invariances géométriques}

Les symétries naturelles des organoïdes (invariance aux translations et rotations) sont généralement traitées via une augmentation des données extensive plutôt que garanties par l'architecture. Cela allonge l'entraînement et ne garantit pas parfaitement l'invariance.

\subsection{Originalité de l'approche proposée}

Notre approche se distingue par plusieurs aspects originaux qui adressent directement les lacunes identifiées.

\subsubsection{Premier framework GNN pour organoïdes 3D}

À notre connaissance, cette thèse propose la première application systématique de Graph Neural Networks géométriques à l'analyse d'organoïdes 3D. Alors que les GNNs sont établis pour l'étude des molécules et des protéines (prédiction de propriétés) et émergent en histopathologie 2D, leur application aux structures biologiques 3D complexes comme les organoïdes est inédite.

\subsubsection{Représentation explicite de la structure relationnelle}

Contrairement aux approches CNN qui considèrent l’organoïde comme une simple grille de voxels, notre méthode repose sur une représentation explicite des cellules individuelles en tant qu’entités discrètes. Nous construisons un graphe dans lequel chaque cellule est identifiée et reliée à ses voisines, ce qui permet de modéliser précisément la topologie et la structure relationnelle de l’organoïde. Cette modélisation rend possible une interprétation fine des mécanismes au niveau cellulaire, ouvrant la voie à une meilleure compréhension des interactions et des comportements collectifs au sein de la structure.

\subsubsection{Équivariance géométrique par construction}

L'utilisation d'architectures EGNN garantit l'invariance aux transformations eucliennes par design architectural, sans nécessiter d'augmentation de données extensive. Cela améliore l'efficacité d'apprentissage et la robustesse.

\subsubsection{Génération synthétique basée sur les processus ponctuels}

Notre approche de génération de données synthétiques via les processus ponctuels sur la sphère est, à notre connaissance, inédite. Elle diffère des approches de data augmentation classiques en créant des organoïdes entièrement nouveaux avec des propriétés statistiques contrôlées, permettant une validation rigoureuse et un pré-entraînement ciblé.

\subsubsection{Pipeline intégré de bout en bout}

Plutôt qu'un ensemble d'outils dispersés, nous proposons un pipeline cohérent, optimisable conjointement, avec des interfaces claires entre les étapes, facilitant l'adoption et la reproduction.

\subsection{Verrous scientifiques et techniques adressés}

Cette thèse s'attaque à trois verrous majeurs, formant les contributions principales.

\subsubsection{Verrou 1 : Représentation adaptée}

\textbf{Question scientifique :} Comment encoder efficacement la structure 3D relationnelle des organoïdes pour l'apprentissage automatique ?

\textbf{Notre réponse :}
\begin{itemize}
    \item Modélisation par graphes géométriques (cellules = nœuds, voisinage = arêtes)
    \item Features multi-modales (position, morphologie)
    \item Compression drastique tout en préservant l'information structurelle
\end{itemize}

\subsubsection{Verrou 2 : Apprentissage avec données limitées}

\textbf{Question scientifique :} Comment entraîner des modèles robustes malgré le manque d'annotations expertes ?

\textbf{Notre réponse :}
\begin{itemize}
    \item Génération de données synthétiques via processus ponctuels
    \item Pré-entraînement sur synthétiques puis fine-tuning sur réels (transfer learning)
\end{itemize}

\subsubsection{Verrou 3 : Robustesse et généralisation}

\textbf{Question scientifique :} Comment assurer la robustesse aux variations expérimentales et la généralisation inter-laboratoires ?

\textbf{Notre réponse :}
\begin{itemize}
    \item Invariances géométriques garanties par architecture équivariante
    \item Normalisation multi-niveau (features, graphes, prédictions)
\end{itemize}

\subsection{Positionnement par rapport aux approches concurrentes}

\subsubsection{Comparaison conceptuelle}

\begin{center}
\begin{tabular}{|l|c|c|c|c|}
\hline
\textbf{Critère} & \textbf{Manuel} & \textbf{Descripteurs} & \textbf{CNN 3D} & \textbf{GNN (Nous)} \\
\hline
Automatisation & - & ++ & +++ & +++ \\
Scalabilité & - & ++ & + & +++ \\
Empreinte mémoire & N/A & +++ & - & +++ \\
Features apprises & - & - & +++ & +++ \\
Invariances géométriques & ++ & + & + & +++ \\
Besoin en données & N/A & + & --- & ++ \\
Capture de relations & - & - & - & +++ \\
\hline
\end{tabular}
\end{center}

Notre approche combine les avantages du deep learning (apprentissage de features), de l'efficacité computationnelle, et de l'expressivité structurelle (capture des relations cellulaires).

\subsubsection{Complémentarité}

Notre approche n'est pas exclusive mais complémentaire :
\begin{itemize}
    \item Elle dépend d'une segmentation cellulaire préalable (Cellpose)
    \item Elle peut être combinée avec des descripteurs handcrafted (features additionnelles)
    \item Elle peut être comparée aux CNN 3D pour validation
    \item Elle peut être évaluée par rapport à l'analyse manuelle
\end{itemize}

\subsection{Questions de recherche principales}

Cette thèse cherche à répondre aux questions suivantes :

\begin{enumerate}
    \item \textbf{Q1 - Représentation} : Les graphes géométriques constituent-ils une représentation efficace des organoïdes pour l'apprentissage automatique ?
    
    \item \textbf{Q2 - Architecture} : Les GNNs équivariants apportent-ils un gain significatif par rapport aux GNNs standards ?
    
    \item \textbf{Q3 - Données synthétiques} : Les données synthétiques générées par processus ponctuels sont-elles réalistes et utiles pour le pré-entraînement ?
    
    \item \textbf{Q4 - Transfer learning} : Le pré-entraînement sur synthétiques améliore-t-il significativement les performances et la data efficiency sur données réelles ?
    
    \item \textbf{Q5 - Généralisation} : L'approche est-elle robuste aux variations expérimentales et généralisable à différents types d'organoïdes ?
\end{enumerate}
