% !TEX root = ../sommaire.tex

\chapter{État de l'art}

\section{Organoïdes : biologie et applications}

\subsection{Définitions et types d'organoïdes}

Les organoïdes sont des structures tridimensionnelles auto-organisées cultivées \textit{in vitro} à partir de cellules souches ou de tissus primaires. Selon la définition de Lancaster et Knoblich~\cite{Lancaster2014}, un organoïde doit satisfaire plusieurs critères : contenir plusieurs types cellulaires de l'organe qu'il représente, présenter une organisation spatiale similaire à celle de l'organe natif, et récapituler au moins certaines fonctions de l'organe.

\subsubsection{Classification par origine cellulaire}

Les organoïdes peuvent être générés à partir de différentes sources cellulaires :

\textbf{Organoïdes dérivés de cellules souches pluripotentes (PSC-derived) :}
\begin{itemize}
    \item Générés à partir d'ESC (cellules souches embryonnaires) ou d'iPSC (cellules souches pluripotentes induites)
    \item Requièrent des protocoles de différenciation dirigée complexes mimant le développement embryonnaire
    \item Permettent de générer des organoïdes de types non accessibles autrement (cerveau, rétine)
    \item Avantage : génération illimitée, manipulation génétique possible
    \item Inconvénient : immaturité fonctionnelle, protocoles longs (plusieurs semaines)
\end{itemize}

\textbf{Organoïdes dérivés de cellules souches adultes (ASC-derived) :}
\begin{itemize}
    \item Générés à partir de cellules souches résidentes dans les tissus adultes (cryptes intestinales, glandes gastriques, etc.)
    \item Croissance en matrice extracellulaire (Matrigel) en présence de facteurs de croissance spécifiques
    \item Maintiennent l'identité tissulaire de l'organe d'origine
    \item Avantage : maturation fonctionnelle supérieure, protocoles plus simples
    \item Inconvénient : accès limité à certains tissus, capacité d'expansion variable
\end{itemize}

\textbf{Organoïdes dérivés de patients (PDO - Patient-Derived Organoids) :}
\begin{itemize}
    \item Sous-catégorie des ASC-derived, générés directement à partir de biopsies de patients
    \item Préservent le profil génétique et épigénétique du patient
    \item Applications majeures en oncologie (tumoroïdes) pour médecine personnalisée
    \item Biobanques constituées pour représenter la diversité des pathologies
\end{itemize}

\subsubsection{Classification par type d'organe}

Différents types d'organoïdes ont été développés avec succès :

\textbf{Organoïdes intestinaux :}
Premier système organoïde développé~\cite{Sato2009}, ils reproduisent l'architecture des villosités intestinales avec cryptes et cellules différenciées (entérocytes, cellules de Paneth, cellules entéroendocrines, cellules caliciformes). Utilisés pour étudier l'homéostasie intestinale, les maladies inflammatoires, les infections, et pour le criblage de drogues.

\textbf{Organoïdes cérébraux :}
Structures complexes mimant le développement cérébral précoce~\cite{Lancaster2013}, contenant différentes régions cérébrales (cortex, hippocampe, plexus choroïde). Applications en neurobiologie du développement, modélisation de maladies neurologiques (microcéphalie, autisme, schizophrénie), et étude de l'impact de pathogènes (virus Zika).

\textbf{Organoïdes hépatiques :}
Reproduisent l'architecture lobulaire du foie avec hépatocytes fonctionnels. Applications en toxicologie (prédiction d'hépatotoxicité), métabolisme de drogues, modélisation d'hépatites virales et de maladies métaboliques.

\textbf{Organoïdes rénaux :}
Contiennent des structures néphroniques avec tubules et podocytes. Utilisés pour modéliser les maladies rénales génétiques et acquises, tester la néphrotoxicité de composés.

\textbf{Organoïdes pulmonaires :}
Modèles des voies aériennes et des alvéoles. Applications pour SARS-CoV-2, mucoviscidose, cancer du poumon.

\textbf{Autres organoïdes :}
Pancréas, rétine, estomac, prostate, glandes salivaires, sein, etc. La diversité croissante reflète l'universalité de l'approche.

\subsection{Mécanismes de formation et auto-organisation}

\subsubsection{Principes d'auto-organisation cellulaire}

La formation d'organoïdes repose sur les capacités intrinsèques d'auto-organisation cellulaire, gouvernées par plusieurs principes fondamentaux :

\textbf{Signalisation morphogénétique :}
Les cellules répondent à des gradients de molécules signal (Wnt, BMP, FGF, Notch) qui guident leur différenciation et leur positionnement spatial. En fournissant exogènement ces facteurs dans des combinaisons appropriées, on peut diriger le développement vers des destins cellulaires spécifiques.

\textbf{Interactions cellule-cellule :}
Les jonctions adhérentes (cadhérines), jonctions serrées, et gap junctions permettent aux cellules de communiquer, de s'organiser en épithéliums polarisés, et de coordonner leur comportement collectif.

\textbf{Interactions cellule-matrice :}
La matrice extracellulaire (ECM), fournie exogènement sous forme de Matrigel ou de matrices synthétiques, fournit un support structural et des signaux biochimiques (intégrines) régulant la forme, la migration et la différenciation cellulaires.

\textbf{Forces mécaniques :}
Les tensions cytosquelettiques, les pressions osmotiques, et les forces contractiles contribuent à façonner l'architecture tridimensionnelle. La formation de lumens résulte notamment de l'apoptose centrale et de la polarisation cellulaire.

\subsubsection{Étapes de développement d'un organoïde}

Le développement typique d'un organoïde intestinal illustre ces principes :
\begin{enumerate}
    \item \textbf{Agrégation initiale} (Jour 0-2) : Les cellules s'agrègent en sphéroïdes compacts dans le Matrigel
    \item \textbf{Polarisation} (Jour 2-4) : Les cellules développent une polarité apico-basale, avec migration des noyaux
    \item \textbf{Formation du lumen} (Jour 4-6) : Apoptose des cellules centrales créant une cavité interne
    \item \textbf{Bourgeonnement} (Jour 6-10) : Formation de bourgeons cryptiques par prolifération asymétrique
    \item \textbf{Différenciation} (Jour 10+) : Émergence de lignages différenciés (cellules absorbantes, sécrétoires)
    \item \textbf{Maturation} (Semaines) : Complexification de l'architecture, maturation fonctionnelle
\end{enumerate}

Cette séquence développementale, reminiscente de l'embryogenèse, se produit de manière largement autonome une fois les conditions initiales établies.

\subsection{Applications en recherche et en médecine}

\subsubsection{Recherche fondamentale}

\textbf{Développement et morphogenèse :}
Les organoïdes permettent d'étudier \textit{in vitro} les mécanismes de formation des organes, précédemment accessibles uniquement via embryologie. Des questions fondamentales sur la régulation génétique, l'auto-organisation, l'émergence de la complexité peuvent être abordées avec manipulations génétiques et imagerie en temps réel.

\textbf{Biologie des cellules souches :}
La niche des cellules souches, leur maintenance, leur différenciation, et leur réponse aux signaux peuvent être étudiées dans un contexte tissulaire 3D physiologique.

\textbf{Interactions hôte-pathogène :}
Les organoïdes fournissent des modèles d'infection plus réalistes que les monocultures 2D. L'infection par virus (rotavirus, norovirus, SARS-CoV-2), bactéries (Helicobacter, Salmonella), ou parasites peut être étudiée avec imagerie en temps réel et analyses fonctionnelles.

\subsubsection{Criblage pharmacologique et drug discovery}

\textbf{Découverte de médicaments :}
Les organoïdes offrent un système intermédiaire entre les cellules 2D (trop simplistes) et les modèles animaux (coûteux, lents, éthiquement problématiques) pour tester l'efficacité de composés. Des plateformes robotisées permettent le criblage de bibliothèques de milliers de molécules.

\textbf{Tests de toxicité :}
Les organoïdes hépatiques et rénaux sont utilisés pour prédire l'hépatotoxicité et la néphrotoxicité de composés en développement, réduisant les échecs tardifs en phases cliniques.

\textbf{Repositionnement de médicaments :}
Tester systématiquement des drogues approuvées sur de nouveaux modèles de maladies pour identifier de nouvelles indications thérapeutiques.

\subsubsection{Médecine personnalisée et applications cliniques}

\textbf{Prédiction de réponse thérapeutique :}
Les organoïdes tumoraux dérivés de patients peuvent être testés contre un panel de chimiothérapies, thérapies ciblées, ou immunothérapies pour prédire \textit{ex vivo} la sensibilité du patient et guider le choix thérapeutique. Plusieurs études pilotes ont démontré une concordance significative entre réponse des organoïdes et réponse clinique des patients.

\textbf{Diagnostic et pronostic :}
Au-delà du traitement, les organoïdes peuvent servir d'outils diagnostiques. La capacité d'expansion d'organoïdes à partir d'une biopsie peut être pronostique. Les profils moléculaires d'organoïdes peuvent compléter les analyses anatomopathologiques traditionnelles.

\textbf{Biobanques d'organoïdes :}
Des biobanques nationales et internationales d'organoïdes (Hubrecht Organoid Technology, Human Cancer Models Initiative) sont constituées pour capturer la diversité génétique et phénotypique des pathologies humaines~\cite{Drost2019}, servant de ressources partagées pour la communauté scientifique.

\subsubsection{Médecine régénérative : promesses futures}

Bien qu'encore largement expérimentale, l'utilisation d'organoïdes pour la transplantation et la réparation tissulaire progresse :
\begin{itemize}
    \item Transplantation d'organoïdes de rétine pour restaurer la vision (essais précliniques)
    \item Organoïdes de foie pour support fonctionnel temporaire en insuffisance hépatique
    \item Organoïdes de peau pour greffes en cas de brûlures étendues
\end{itemize}

Les défis majeurs incluent la vascularisation (organoïdes > 1mm nécessitent apport sanguin), l'innervation, et l'intégration avec l'hôte.

\subsection{Biomarqueurs et phénotypes d'intérêt}

L'identification et la quantification de phénotypes dans les organoïdes reposent sur l'évaluation de multiples biomarqueurs.

\subsubsection{Marqueurs de prolifération et de mort cellulaire}

\begin{itemize}
    \item \textbf{Ki67} : Protéine nucléaire présente dans toutes les phases actives du cycle cellulaire (G1, S, G2, M), absente en phase quiescente (G0). Marqueur de référence de l'activité proliférative.
    \item \textbf{EdU/BrdU} : Analogues de thymidine incorporés pendant la phase S (réplication de l'ADN), marquant spécifiquement les cellules en division active.
    \item \textbf{Caspase-3 clivée} : Marqueur d'apoptose (mort cellulaire programmée), augmenté dans les régions centrales d'organoïdes de grande taille où survient une hypoxie.
    \item \textbf{Annexin V} : Détecte l'externalisation de phosphatidylsérine, événement précoce de l'apoptose.
\end{itemize}

\subsubsection{Marqueurs de différenciation et de lignage}

\begin{itemize}
    \item \textbf{Marqueurs de cellules souches} : LGR5, SOX2, OCT4, NANOG selon le type d'organoïde
    \item \textbf{Marqueurs de différenciation} : Spécifiques au lignage et au type cellulaire
    \begin{itemize}
        \item Intestin : Lysozyme (cellules de Paneth), Chromogranine A (cellules entéroendocrines), Mucine 2 (cellules caliciformes)
        \item Cerveau : DCX (neurones immatures), GFAP (astrocytes), MAP2 (neurones matures)
        \item Foie : Albumine (hépatocytes), CK19 (cholangiocytes)
    \end{itemize}
\end{itemize}

\subsubsection{Marqueurs fonctionnels}

\begin{itemize}
    \item \textbf{Métabolisme} : Activité enzymatique (CYP450 dans foie), transport (pompes ABC)
    \item \textbf{Sécrétion} : Production de protéines spécifiques (albumine, mucus, hormones)
    \item \textbf{Activité électrophysiologique} : Pour organoïdes neuronaux (calcium imaging)
    \item \textbf{Barrière épithéliale} : Perméabilité, expression de jonctions serrées
\end{itemize}

\subsubsection{Phénotypes composites d'intérêt}

Les phénotypes biologiquement significatifs combinent souvent plusieurs marqueurs et caractéristiques morphologiques :
\begin{itemize}
    \item \textbf{Maturation} : Ratio cellules souches/différenciées, expression de marqueurs tardifs, fonctionnalité
    \item \textbf{Polarisation} : Orientation correcte apico-basale, localisation de marqueurs de polarité
    \item \textbf{Réponse à traitement} : Changements de prolifération, apoptose, morphologie post-traitement
    \item \textbf{Pathologique vs sain} : Signatures distinguant organoïdes normaux de tumoroïdes
\end{itemize}

L'identification automatisée et objective de ces phénotypes constitue un besoin crucial pour l'exploitation des organoïdes.

\section{Analyse d'images biomédicales 3D}

\subsection{Modalités d'imagerie pour organoïdes}

\subsubsection{Microscopie confocale}

La microscopie confocale à balayage laser (CLSM) est la modalité la plus couramment utilisée pour l'imagerie d'organoïdes~\cite{Litjens2017}.

\textbf{Principe :}
Un laser excite la fluorescence point par point, un trou d'épingle (\textit{pinhole}) rejette la lumière hors-foyer, construisant une image optiquement sectionnée. Le balayage du faisceau laser et le déplacement Z du plan focal permettent l'acquisition de stacks 3D.

\textbf{Avantages :}
\begin{itemize}
    \item Haute résolution latérale (200-300 nm) et axiale (500-800 nm)
    \item Rejet efficace de la lumière hors-foyer
    \item Imagerie multicanale (jusqu'à 4-6 fluorophores simultanés)
    \item Disponibilité large dans les laboratoires
\end{itemize}

\textbf{Limitations :}
\begin{itemize}
    \item Acquisition lente (plusieurs minutes par organoïde)
    \item Photoblanchiment important du fait de l'exposition répétée
    \item Phototoxicité limitant l'imagerie live prolongée
    \item Pénétration limitée en profondeur (< 100-150 μm en tissu dense)
\end{itemize}

\subsubsection{Microscopie light-sheet (LSFM)}

La microscopie à feuillet de lumière illumine l'échantillon latéralement avec une nappe de lumière fine, tandis que la détection se fait perpendiculairement~\cite{Huisken2004}.

\textbf{Avantages :}
\begin{itemize}
    \item Vitesse d'acquisition élevée (plusieurs images par seconde)
    \item Photoblanchiment et phototoxicité minimaux
    \item Pénétration profonde après clarification optique
    \item Idéale pour imagerie time-lapse prolongée
    \item Imagerie de grands volumes (plusieurs mm$^3$)
\end{itemize}

\textbf{Limitations :}
\begin{itemize}
    \item Équipements spécialisés, moins répandus
    \item Nécessite clarification optique pour tissus denses
    \item Résolution légèrement inférieure à la confocale
\end{itemize}

\subsubsection{Microscopie multiphoton}

L'excitation multiphoton utilise des impulsions laser infrarouge intenses pour exciter les fluorophores via absorption simultanée de deux photons.

\textbf{Avantages :}
\begin{itemize}
    \item Pénétration profonde (jusqu'à 1 mm) grâce à la longueur d'onde infrarouge
    \item Réduction du photoblanchiment (excitation confinée au point focal)
    \item Imagerie \textit{in vivo} possible
\end{itemize}

\textbf{Limitations :}
\begin{itemize}
    \item Lasers coûteux (femtoseconde Ti:Sapphire)
    \item Acquisition plus lente que confocale standard
    \item Fluorophores optimisés pour multiphoton requis
\end{itemize}

\subsubsection{Techniques de clarification optique}

Pour améliorer la pénétration de la lumière dans les tissus épais, diverses techniques de clarification ont été développées~\cite{Chung2010} :
\begin{itemize}
    \item \textbf{CLARITY} : Remplacement des lipides par hydrogel polyacrylamide, rendant l'échantillon transparent
    \item \textbf{iDISCO} : Délipidation par solvants organiques
    \item \textbf{CUBIC} : Délipidation aqueuse plus douce
    \item \textbf{Expansion microscopy} : Expansion physique de l'échantillon (4x) pour augmenter virtuellement la résolution~\cite{Chen2015}
\end{itemize}

Ces techniques, appliquées aux organoïdes, permettent l'imagerie de leur structure interne complète.

\subsection{Défis spécifiques de l'imagerie d'organoïdes}

\subsubsection{Résolution, bruit et artefacts}

\textbf{Diffusion de la lumière :}
Dans les tissus biologiques, la diffusion (scattering) de la lumière dégrade la résolution et le contraste avec la profondeur. Les structures périphériques sont mieux résolues que le cœur de l'organoïde.

\textbf{Hétérogénéité d'illumination :}
L'atténuation de la lumière crée des gradients d'intensité non-uniformes, compliquant la segmentation automatique. Les cellules profondes apparaissent plus sombres que les cellules superficielles.

\textbf{Aberrations optiques :}
Les différences d'indice de réfraction entre milieu de montage, Matrigel et tissu créent des aberrations sphériques et chromatiques dégradant la qualité d'image.

\textbf{Bruit photonique :}
À faibles niveaux de lumière (imagerie live pour réduire phototoxicité), le bruit de photons (Poisson) devient significatif, dégradant le rapport signal/bruit.

\subsubsection{Variabilité inter-acquisitions}

Les conditions d'imagerie varient entre expériences :
\begin{itemize}
    \item Puissance laser, gain du détecteur, temps d'exposition
    \item Performances optiques (alignement, vieillissement des lasers)
    \item Orientation de l'échantillon (organoïdes non fixés peuvent bouger)
    \item Protocoles de marquage (efficacité de pénétration des anticorps, photoblanchiment différentiel)
\end{itemize}

Ces variations rendent difficile le développement de méthodes d'analyse universellement applicables sans stratégies de normalisation robustes.

\subsection{Contraintes computationnelles}

\subsubsection{Volumes de données}

Un organoïde imagé en haute résolution (objectif 40×, voxel 0.3×0.3×1 μm) génère typiquement :
\begin{itemize}
    \item Taille : 2048×2048×150-300 slices
    \item Canaux : 3-4 fluorophores (DAPI + marqueurs)
    \item Profondeur : 16 bits par voxel
    \item \textbf{Volume total} : 2-4 Go par organoïde
\end{itemize}

Pour une étude de criblage pharmacologique testant 100 composés × 10 concentrations × 5 réplicats × 3 temps = 15,000 organoïdes, le volume total atteint 30-60 To.

\subsubsection{Défis de traitement}

Le traitement de ces volumes pose plusieurs défis :
\begin{itemize}
    \item \textbf{Mémoire} : Charger une image complète en mémoire peut dépasser la RAM disponible (64-128 Go typiques)
    \item \textbf{Temps de calcul} : L'analyse naïve voxel-par-voxel est prohibitivement lente
    \item \textbf{Parallélisation} : Nécessité de stratégies de traitement par blocs (tiling) ou de streaming
    \item \textbf{Stockage} : Coûts de stockage et de backup importants, nécessité de compression
\end{itemize}

Ces contraintes motivent le développement de représentations compactes et efficaces, comme les graphes cellulaires proposés dans cette thèse.

\section{Méthodes d'analyse existantes}

\subsection{Analyse manuelle : référence mais non scalable}

\subsubsection{Protocoles d'analyse experte}

L'analyse manuelle typique implique :
\begin{enumerate}
    \item Visualisation 3D de l'organoïde (rendus volumiques, projections maximum intensity)
    \item Évaluation qualitative de la morphologie globale
    \item Inspection de sections 2D à différentes profondeurs
    \item Quantification semi-manuelle de marqueurs (comptage de cellules positives)
    \item Classification selon des critères prédéfinis ou une échelle ordinale
\end{enumerate}

\subsubsection{Performances et limites quantifiées}

Études sur l'accord inter-annotateurs pour classification d'organoïdes :
\begin{itemize}
    \item Phénotypes binaires (sain/pathologique) : Cohen's κ = 0.7-0.85 (accord substantiel)
    \item Phénotypes multi-classes subtils : κ = 0.5-0.7 (accord modéré)
    \item Quantifications continues (taille, intensités) : coefficient de corrélation intra-classe ICC = 0.6-0.8
\end{itemize}

Ces valeurs, bien que respectables, révèlent une variabilité non négligeable même entre experts, soulignant la subjectivité inhérente et la difficulté de ces tâches.

\subsubsection{Coûts et contraintes pratiques}

\textbf{Coûts temporels :}
\begin{itemize}
    \item Analyse simple : 5-10 min/organoïde
    \item Analyse détaillée : 15-30 min/organoïde
    \item Pour 1000 organoïdes : 250-500 heures = 1-2 mois temps plein
\end{itemize}

\textbf{Coûts financiers :}
À 50€/heure de temps expert, l'analyse de 1000 organoïdes coûte 12,500-25,000€, motivant l'automatisation.

\subsection{Segmentation cellulaire : état de l'art}

La segmentation automatique des cellules constitue une étape critique précédant toute analyse quantitative. Plusieurs approches ont été développées.

\subsubsection{Méthodes classiques de segmentation}

\textbf{Seuillage et détection de blobs :}
Les approches les plus simples appliquent un seuil global ou adaptatif sur le canal nucléaire (DAPI), puis détectent les régions connectées. Limitations : fusion de noyaux proches, sensibilité au choix de seuil.

\textbf{Watershed :}
L'algorithme de watershed traite l'image comme une surface topographique où les intensités représentent l'altitude. Les minima locaux sont inondés progressivement jusqu'à ce que les bassins versants se rencontrent, définissant les frontières de segmentation. Problème majeur : sur-segmentation nécessitant un post-traitement (détection de seeds, marker-controlled watershed).

\textbf{Active contours et level sets :}
Des contours évoluent sous l'influence de forces internes (régularité) et externes (gradients d'intensité) pour épouser les frontières cellulaires. Méthodes élégantes mais lentes en 3D et sensibles à l'initialisation.

\subsubsection{Approches par deep learning}

\textbf{U-Net et variantes :}
L'architecture U-Net~\cite{Ronneberger2015}, introduite pour segmentation biomédicale 2D, a été étendue en 3D~\cite{Cicek2016}. L'architecture encoder-decoder avec skip connections permet la segmentation sémantique dense. Limitations : nécessite annotations pixel-level coûteuses, empreinte mémoire importante en 3D.

\textbf{Mask R-CNN et détection d'instances :}
Détection et segmentation simultanées d'instances d'objets. Applicable aux noyaux cellulaires mais difficultés pour gérer les chevauchements en 3D.

\textbf{StarDist :}
Approche innovante représentant chaque cellule par son centroïde et un champ de distances radiales (star-convex polygons)~\cite{Schmidt2018}. Très performant pour noyaux de forme convexe. Implémentation 2D et 3D disponible. Limitation : formes non-convexes mal gérées.

\textbf{Cellpose :}
État de l'art actuel~\cite{Stringer2021}, basé sur la prédiction de champs de gradients où chaque pixel/voxel "pointe" vers le centre de sa cellule. Le suivi de ces gradients permet de regrouper les pixels en cellules. Avantages majeurs :
\begin{itemize}
    \item Robuste aux variations de taille cellulaire
    \item Gère bien les morphologies irrégulières et les cellules denses
    \item Version 3D efficace
    \item Modèles pré-entraînés généralist es performants
    \item Possibilité de fine-tuning sur données spécifiques
\end{itemize}

Cellpose représente actuellement le meilleur compromis précision/généralisation pour la segmentation de noyaux et de cellules en 3D.

\subsubsection{Évaluation et benchmarking}

Les méthodes de segmentation sont typiquement évaluées via :
\begin{itemize}
    \item \textbf{Métriques pixel-level} : Dice coefficient, Intersection over Union (IoU)
    \item \textbf{Métriques objet-level} : Précision, rappel, F1-score sur détection d'instances
    \item \textbf{Average Precision (AP)} : Métrique standard des défis de segmentation (issu de détection d'objets)
    \item \textbf{Segmentation Covering} : Mesure de recouvrement entre segmentation prédite et vérité terrain
\end{itemize}

Les compétitions internationales (Cell Tracking Challenge~\cite{Ulman2017}, Data Science Bowl~\cite{Caicedo2019}) ont poussé l'état de l'art avec des datasets de référence annotés.

\subsection{Approches par vision par ordinateur classique}

\subsubsection{Descripteurs de texture}

Les organoïdes présentent des textures caractéristiques (distributions d'intensités, arrangements spatiaux) qu capturées par divers descripteurs.

\textbf{Matrices de co-occurrence de Haralick :}
Calculent des statistiques de second ordre sur les paires de pixels à distance et orientation données. Features extraites : contraste, corrélation, énergie, homogénéité, entropie. Extension 3D possible mais coûteuse. Limitation : perte de l'information spatiale globale, descripteurs génériques peu spécifiques.

\textbf{Local Binary Patterns (LBP) :}
Codent localement les relations d'intensité entre pixel central et voisins. LBP 3D capturent des micro-textures tridimensionnelles. Robustes aux variations d'illumination mais sensibles au bruit.

\textbf{Filtres de Gabor :}
Convolutions avec des noyaux orientés à différentes fréquences et orientations. Capturent des patterns directionnels et des textures répétitives. Banques de filtres 3D computationnellement coûteuses.

\subsubsection{Descripteurs géométriques et morphologiques}

\textbf{Moments géométriques :}
Moments d'ordre 0 (volume), 1 (centroïde), 2 (matrice d'inertie, axes principaux), 3+ (asymétrie, kurtosis). Les moments de Hu invariants par transformation peuvent caractériser la forme.

\textbf{Descripteurs de forme :}
\begin{itemize}
    \item Sphéricité : $\Psi = \frac{\pi^{1/3}(6V)^{2/3}}{S}$ où $V$ est le volume et $S$ la surface
    \item Excentricité : Rapport des axes principaux
    \item Compacité, convexité, solidité
    \item Descripteurs de Fourier de la surface
\end{itemize}

\textbf{Analyse de distribution spatiale :}
Gradients radiaux d'intensité (du centre vers périphérie), profils d'intensité le long d'axes, moments spatiaux d'ordre supérieur.

\subsubsection{Machine learning classique sur descripteurs}

Une fois les features extraites, des algorithmes de ML classique sont appliqués :

\textbf{Random Forest :}
Ensembles d'arbres de décision, robustes, gèrent bien les features hétérogènes et les non-linéarités. Fournissent des importances de features pour interprétabilité.

\textbf{Support Vector Machines (SVM) :}
Avec noyaux non-linéaires (RBF), performants pour classification avec données limitées. Sensibles au scaling des features.

\textbf{Gradient Boosting (XGBoost, LightGBM) :}
État de l'art pour données tabulaires, souvent supérieurs à Random Forest. Nécessitent tuning d'hyperparamètres.

\textbf{Résultats typiques :}
Sur des tâches de classification d'organoïdes avec descripteurs handcrafted, les performances atteignent typiquement 70-85\% d'accuracy selon la difficulté du problème et la qualité des descripteurs.

\textbf{Limitations principales :}
\begin{itemize}
    \item Plafond de performance limité par l'expressivité des features
    \item Nécessité d'expertise domaine pour feature engineering
    \item Perte d'information relationnelle (relations entre cellules non capturées)
    \item Peu de généralisation à d'autres types d'organoïdes
\end{itemize}

\subsection{Deep learning pour images biomédicales}

\subsubsection{CNN 2D : approches par slices}

\textbf{Max/Mean intensity projections :}
Projeter le volume 3D en 2D (maximum ou moyenne selon Z) puis appliquer un CNN 2D standard (ResNet, EfficientNet). Perte massive d'information 3D, mais computationnellement abordable.

\textbf{Multi-slice analysis :}
Extraire plusieurs slices 2D à différentes profondeurs, classifier chaque slice, puis agréger les prédictions (vote majoritaire, moyenne). Meilleur que projection mais cohérence spatiale 3D non exploitée.

\textbf{2.5D approaches :}
Créer des pseudo-images RGB en stackant 3 slices adjacentes, exploitant l'architecture CNN 2D standard. Compromis entre information 3D et efficacité.

\subsubsection{CNN 3D : extension naturelle mais coûteuse}

Les CNN 3D étendent les opérations de convolution et pooling à trois dimensions.

\textbf{Architectures classiques 3D :}
\begin{itemize}
    \item \textbf{3D U-Net} : Segmentation sémantique dense en 3D, encoder-decoder avec skip connections
    \item \textbf{V-Net} : Variante avec residual connections et convolutions 3D
    \item \textbf{ResNet 3D, DenseNet 3D} : Adaptations des architectures 2D célèbres
\end{itemize}

\textbf{Contraintes pratiques :}
Pour fitter en mémoire GPU (16-32 Go), il faut :
\begin{itemize}
    \item Downsampler drastiquement (64×64×64 ou 128×128×128 max)
    \item Traiter par patches/crops avec recombinaison
    \item Réduire la profondeur du réseau
    \item Limiter le batch size (souvent 1-2 échantillons)
\end{itemize}

Le downsampling détruit les détails fins (cellules individuelles non résolues), limitant la capacité à capturer l'information biologique subtile.

\textbf{Résultats et limitations :}
Bien que les CNN 3D obtiennent de bonnes performances sur certaines tâches de classification d'organoïdes (85-90\%), ils :
\begin{itemize}
    \item Nécessitent de larges datasets annotés (milliers d'exemples)
    \item Sont sensibles aux variations d'acquisition (domain shift)
    \item Manquent d'interprétabilité (boîte noire)
    \item Requièrent augmentation extensive pour invariances géométriques
\end{itemize}

\subsection{Approches basées graphes en histopathologie}

\subsubsection{Contexte : graphes cellulaires en pathologie numérique}

Quelques travaux pionniers ont exploré l'utilisation de graphes pour l'analyse de tissus en histopathologie 2D~\cite{Zhou2019,Jaume2021,Pati2022}, anticipant notre approche.

\textbf{Cell graphs pour classification de cancers :}
Les cellules dans une coupe histologique 2D sont représentées comme nœuds d'un graphe, connectées selon la proximité spatiale. Les features incluent morphologie nucléaire, texture, intensité de coloration. Des GNNs sont entraînés pour prédire le grade tumoral, le sous-type histologique, ou le pronostic.

\textbf{Tissue graphs multi-échelles :}
Représentations hiérarchiques où les nœuds peuvent être des cellules individuelles, des régions tissulaires, ou des glandes entières. Capture d'informations multi-échelles pertinentes pour le diagnostic.

\textbf{Spatial transcriptomics :}
Avec les technologies de transcriptomique spatiale, chaque "spot" (région de ~50 μm) est un nœud avec comme features le profil d'expression génique. Les GNNs intègrent l'information spatiale pour identifier des niches cellulaires, prédire des états cellulaires.

\subsubsection{Résultats prometteurs}

Ces approches ont démontré :
\begin{itemize}
    \item Performances supérieures ou comparables aux CNN sur certaines tâches
    \item Meilleure interprétabilité (cellules/régions importantes identifiables)
    \item Robustesse aux variations de taille d'image
    \item Capacité à capturer des patterns topologiques (clustering, arrangement spatial)
\end{itemize}

\subsubsection{Limitations et extension aux organoïdes 3D}

Cependant, ces travaux présentent des limites importantes :
\begin{itemize}
    \item \textbf{Limitation 2D} : Les coupes histologiques sont intrinsèquement 2D, perdant l'information 3D cruciale des organoïdes
    \item \textbf{Tissus plans} : Les approches sont conçues pour des architectures planaires, pas pour des structures sphéroïdales 3D
    \item \textbf{Features simples} : Les graphes utilisés n'exploitent généralement pas les coordonnées spatiales 3D explicitement
    \item \textbf{Pas d'équivariance géométrique} : Les architectures ne respectent pas les symétries 3D naturelles
\end{itemize}

L'extension de ces approches aux organoïdes 3D avec graphes géométriques équivariants constitue une contribution originale de notre travail.

\section{Positionnement de la thèse}

\subsection{Lacunes identifiées dans la littérature}

Notre revue de la littérature révèle plusieurs lacunes majeures :

\subsubsection{Absence de méthodes automatisées spécifiques aux organoïdes 3D}

À notre connaissance, aucune méthode publiée ne propose un framework complet et automatisé spécifiquement conçu pour l'analyse d'organoïdes 3D. Les outils existants sont soit :
\begin{itemize}
    \item Génériques (ImageJ, CellProfiler) nécessitant une expertise utilisateur importante
    \item Spécifiques à d'autres types de données (histopathologie 2D, cultures 2D)
    \item Fragmentés (segmentation séparée de l'analyse)
    \item Non validés rigoureusement sur organoïdes
\end{itemize}

\subsubsection{Sous-exploitation de la structure relationnelle}

Les méthodes existantes traitent majoritairement les organoïdes comme des images, sans modéliser explicitement le réseau d'interactions cellulaires qui gouverne leur comportement. Les CNN opèrent sur des voxels, les descripteurs calculent des statistiques globales, mais la topologie du graphe cellulaire n'est pas exploitée.

\subsubsection{Manque de solutions au problème de données limitées}

La rareté de données annotées est un verrou reconnu mais peu adressé. Aucune approche de génération de données synthétiques spécifique aux organoïdes n'a été proposée. Les stratégies de transfer learning ou de few-shot learning pour ce domaine sont inexistantes.

\subsubsection{Absence de prise en compte des invariances géométriques}

Les symétries naturelles des organoïdes (invariance aux rotations) sont généralement traitées via augmentation de données extensive plutôt que garanties architecturalement. Cela allonge l'entraînement et ne garantit pas parfaitement l'invariance.

\subsection{Originalité de l'approche proposée}

Notre approche se distingue par plusieurs aspects originaux qui adressent directement les lacunes identifiées.

\subsubsection{Premier framework GNN pour organoïdes 3D}

À notre connaissance, cette thèse propose la première application systématique de Graph Neural Networks géométriques à l'analyse d'organoïdes 3D. Alors que les GNNs sont établis en chimie (prédiction de propriétés moléculaires) et émergent en histopathologie 2D, leur application aux structures biologiques 3D complexes comme les organoïdes est inédite.

\subsubsection{Représentation explicite de la structure relationnelle}

Contrairement aux CNN qui traitent l'organoïde comme une grille de voxels, notre approche :
\begin{itemize}
    \item Identifie explicitement les cellules individuelles comme entités discrètes
    \item Modélise leurs relations de voisinage via un graphe
    \item Capture la topologie et la structure relationnelle
    \item Permet l'interprétation au niveau cellulaire
\end{itemize}

\subsubsection{Équivariance géométrique par construction}

L'utilisation d'architectures EGNN garantit l'invariance aux transformations eucliennes par design architectural, sans nécessiter d'augmentation de données extensive. Cela améliore l'efficacité d'apprentissage et la robustesse.

\subsubsection{Génération synthétique basée processus ponctuels}

Notre approche de génération de données synthétiques via processus ponctuels sur la sphère est, à notre connaissance, inédite. Elle diffère des approches de data augmentation classiques en créant des organoïdes entièrement nouveaux avec propriétés statistiques contrôlées, permettant une validation rigoureuse et un pré-entraînement ciblé.

\subsubsection{Pipeline intégré de bout en bout}

Plutôt qu'un ensemble d'outils dispersés, nous proposons un pipeline cohérent, optimisable conjointement, avec interfaces claires entre étapes, facilitant l'adoption et la reproduction.

\subsection{Verrous scientifiques et techniques adressés}

Cette thèse s'attaque à quatre verrous majeurs, formant les contributions principales.

\subsubsection{Verrou 1 : Représentation adaptée}

\textbf{Question scientifique :} Comment encoder efficacement la structure 3D relationnelle des organoïdes pour l'apprentissage automatique ?

\textbf{Notre réponse :}
\begin{itemize}
    \item Modélisation par graphes géométriques (cellules = nœuds, voisinage = arêtes)
    \item Features multi-modales (position, morphologie, intensités)
    \item Compression drastique tout en préservant l'information structurelle
    \item Validation : comparaison performances graphes vs images brutes vs descripteurs
\end{itemize}

\subsubsection{Verrou 2 : Apprentissage avec données limitées}

\textbf{Question scientifique :} Comment entraîner des modèles robustes malgré le manque d'annotations expertes ?

\textbf{Notre réponse :}
\begin{itemize}
    \item Génération de données synthétiques via processus ponctuels
    \item Pré-entraînement sur synthétiques puis fine-tuning sur réels (transfer learning)
    \item Validation : courbes d'apprentissage, data efficiency, analyse ablative
\end{itemize}

\subsubsection{Verrou 3 : Interprétabilité biologiquement significative}

\textbf{Question scientifique :} Comment rendre les prédictions exploitables par les biologistes et identifier les mécanismes sous-jacents ?

\textbf{Notre réponse :}
\begin{itemize}
    \item Mécanismes d'attention identifiant cellules importantes
    \item Visualisation 3D des contributions cellulaires
    \item Corrélation avec biomarqueurs biologiques connus
    \item Validation qualitative avec experts
\end{itemize}

\subsubsection{Verrou 4 : Robustesse et généralisation}

\textbf{Question scientifique :} Comment assurer la robustesse aux variations expérimentales et la généralisation inter-laboratoires ?

\textbf{Notre réponse :}
\begin{itemize}
    \item Invariances géométriques garanties par architecture équivariante
    \item Normalisation multi-niveau (features, graphes, prédictions)
    \item Évaluation de généralisation inter-batches expérimentaux
    \item Validation croisée rigoureuse
\end{itemize}

\subsection{Positionnement par rapport aux approches concurrentes}

\subsubsection{Comparaison conceptuelle}

\begin{center}
\begin{tabular}{|l|c|c|c|c|}
\hline
\textbf{Critère} & \textbf{Manuel} & \textbf{Descripteurs} & \textbf{CNN 3D} & \textbf{GNN (Nous)} \\
\hline
Automatisation & - & ++ & +++ & +++ \\
Scalabilité & - & ++ & + & +++ \\
Empreinte mémoire & N/A & +++ & - & +++ \\
Interprétabilité & +++ & ++ & - & ++ \\
Features apprises & - & - & +++ & +++ \\
Invariances géométriques & ++ & + & + & +++ \\
Besoin en données & N/A & + & --- & ++ \\
Capture de relations & - & - & - & +++ \\
\hline
\end{tabular}
\end{center}

Notre approche combine les avantages du deep learning (apprentissage de features), de l'efficacité computationnelle, et de l'expressivité structurelle (capture des relations cellulaires).

\subsubsection{Complémentarité}

Notre approche n'est pas exclusive mais complémentaire :
\begin{itemize}
    \item Elle \textbf{dépend} d'une segmentation cellulaire préalable (Cellpose)
    \item Elle peut être \textbf{combinée} avec des descripteurs handcrafted (features additionnelles)
    \item Elle peut être \textbf{comparée} aux CNN 3D pour validation
    \item Elle sera \textbf{évaluée} par rapport à l'analyse manuelle (gold standard)
\end{itemize}

\subsection{Questions de recherche principales}

Cette thèse cherche à répondre aux questions suivantes :

\begin{enumerate}
    \item \textbf{Q1 - Représentation} : Les graphes géométriques constituent-ils une représentation efficace des organoïdes pour l'apprentissage automatique ?
    
    \item \textbf{Q2 - Architecture} : Les GNNs équivariants apportent-ils un gain significatif par rapport aux GNNs standards et aux CNN 3D ?
    
    \item \textbf{Q3 - Données synthétiques} : Les données synthétiques générées par processus ponctuels sont-elles réalistes et utiles pour le pré-entraînement ?
    
    \item \textbf{Q4 - Transfer learning} : Le pré-entraînement sur synthétiques améliore-t-il significativement les performances et la data efficiency sur données réelles ?
    
    \item \textbf{Q5 - Interprétabilité} : Les cellules et patterns identifiés comme importants par le modèle correspondent-ils à des mécanismes biologiques connus ?
    
    \item \textbf{Q6 - Généralisation} : L'approche est-elle robuste aux variations expérimentales et généralisable à différents types d'organoïdes ?
\end{enumerate}

Le Chapitre 5 (Résultats) répondra empiriquement à chacune de ces questions via des expériences ciblées.
