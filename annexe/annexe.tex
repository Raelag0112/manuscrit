% !TEX root = ../sommaire.tex

\chapter{Fondamentaux du Deep Learning}

Ce premier chapitre d'annexe fournit les bases du deep learning nécessaires à la compréhension de notre travail, sans supposer de connaissances préalables approfondies.

\section{Histoire et évolutions majeures}

\subsection{Les origines : perceptrons et réseaux multicouches}

Le concept de neurone artificiel remonte aux années 1940 avec le modèle McCulloch-Pitts. Le perceptron de Rosenblatt (1958) marque la première implémentation matérielle d'un réseau de neurones. Les perceptrons multicouches (MLP), introduits dans les années 1980, avec l'algorithme de rétropropagation du gradient (Rumelhart et al., 1986), permettent pour la première fois d'entraîner des réseaux profonds.

\subsection{Premier hiver de l'IA et renaissance}

Les limitations computationnelles et théoriques (problème du XOR, vanishing gradient) provoquent un "hiver de l'IA" dans les années 1990. La renaissance vient avec :
\begin{itemize}
    \item \textbf{LeNet (LeCun et al., 1998)} : Premier CNN appliqué avec succès à la reconnaissance de chiffres manuscrits
    \item \textbf{Amélioration hardware} : GPUs permettant parallélisation massive
    \item \textbf{Grandes données} : Émergence d'ImageNet et autres datasets massifs
\end{itemize}

\subsection{Révolution AlexNet (2012)}

AlexNet~\cite{Krizhevsky2012} marque le début de l'ère moderne du deep learning en remportant ImageNet avec une marge sans précédent. Innovations clés :
\begin{itemize}
    \item Architecture profonde (8 couches)
    \item Utilisation de ReLU (au lieu de sigmoïde/tanh)
    \item Dropout pour régularisation
    \item Data augmentation extensive
    \item Entraînement sur GPU (2 GTX 580)
\end{itemize}

\subsection{Ère des architectures très profondes}

\textbf{VGGNet (2014)} : Démontre que la profondeur (16-19 couches) améliore performances, mais au prix du nombre de paramètres.

\textbf{ResNet (2015)}~\cite{He2016} : Révolution avec skip connections (residual connections) permettant d'entraîner réseaux de 50, 101, même 152 couches sans dégradation. L'idée : apprendre des résidus $F(x) = H(x) - x$ plutôt que la fonction complète.

\textbf{DenseNet, MobileNet, EfficientNet} : Architectures optimisant le compromis performance/efficacité.

\subsection{Révolution Transformer (2017)}

Le mécanisme d'attention~\cite{Vaswani2017}, introduit pour NLP (traduction), révolutionne le deep learning :
\begin{itemize}
    \item Remplace récurrence par attention parallélisable
    \item Capture dépendances à longue distance
    \item Scalabilité à très grandes données
\end{itemize}

Extensions : BERT (2018), GPT (2018-2023), Vision Transformers (2020).

\subsection{Modèles de fondation et ère actuelle}

GPT-3 (2020), DALL-E, Stable Diffusion marquent l'émergence de modèles massifs pré-entraînés adaptables à multiples tâches. Le paradigme "pre-train + fine-tune" devient dominant.

\section{Architectures classiques}

\subsection{Perceptrons multicouches (MLP)}

\subsubsection{Architecture}

Un MLP est une séquence de couches fully-connected :
\[
\mathbf{h}^{(l+1)} = \sigma\left(\mathbf{W}^{(l)}\mathbf{h}^{(l)} + \mathbf{b}^{(l)}\right)
\]

où $\mathbf{W}^{(l)}$ sont les poids, $\mathbf{b}^{(l)}$ les biais, $\sigma$ la fonction d'activation.

\subsubsection{Fonctions d'activation}

\textbf{Sigmoïde} : $\sigma(x) = \frac{1}{1 + e^{-x}}$
\begin{itemize}
    \item Sort dans $[0, 1]$, interprétable comme probabilité
    \item Problème : saturation $\rightarrow$ vanishing gradient
\end{itemize}

\textbf{Tanh} : $\tanh(x) = \frac{e^x - e^{-x}}{e^x + e^{-x}}$
\begin{itemize}
    \item Sort dans $[-1, 1]$, centré en 0
    \item Moins de saturation que sigmoïde mais persiste
\end{itemize}

\textbf{ReLU} : $\text{ReLU}(x) = \max(0, x)$
\begin{itemize}
    \item Résout vanishing gradient (gradient = 0 ou 1)
    \item Sparse activation
    \item Problème : dying ReLU (neurones morts)
\end{itemize}

\textbf{Variantes} : Leaky ReLU, PReLU, ELU, GELU, Swish/SiLU

\subsubsection{Rétropropagation}

L'algorithme de rétropropagation calcule efficacement les gradients $\frac{\partial \mathcal{L}}{\partial \mathbf{W}^{(l)}}$ via la règle de dérivation en chaîne.

Pour une couche $l$ :
\[
\frac{\partial \mathcal{L}}{\partial \mathbf{W}^{(l)}} = \frac{\partial \mathcal{L}}{\partial \mathbf{h}^{(l+1)}} \frac{\partial \mathbf{h}^{(l+1)}}{\partial \mathbf{W}^{(l)}} = \delta^{(l+1)} (\mathbf{h}^{(l)})^T
\]

où $\delta^{(l+1)}$ est l'erreur rétropropagée.

\subsection{Réseaux de neurones convolutifs (CNN)}

\subsubsection{Motivation}

Les images possèdent trois propriétés exploitables :
\begin{enumerate}
    \item \textbf{Localité spatiale} : Pixels voisins sont corrélés
    \item \textbf{Stationnarité} : Même features visuelles à différentes positions
    \item \textbf{Hiérarchie} : Features simples (edges) → complexes (objets)
\end{enumerate}

Les CNN exploitent ces propriétés via trois mécanismes : convolution, pooling, architecture hiérarchique.

\subsubsection{Opération de convolution}

Une convolution 2D applique un filtre $\mathbf{K}$ de taille $k \times k$ :
\[
(\mathbf{I} * \mathbf{K})_{i,j} = \sum_{m=0}^{k-1} \sum_{n=0}^{k-1} \mathbf{I}_{i+m, j+n} \cdot \mathbf{K}_{m,n}
\]

\textbf{Propriétés} :
\begin{itemize}
    \item Partage de poids : même filtre appliqué partout
    \item Champ réceptif local : connexions limitées
    \item Équivariance par translation : $f(\text{shift}(x)) = \text{shift}(f(x))$
\end{itemize}

\subsubsection{Pooling}

Le pooling réduit la dimensionnalité :
\begin{itemize}
    \item \textbf{Max pooling} : $p_{i,j} = \max_{(m,n) \in \text{window}} x_{i+m, j+n}$
    \item \textbf{Average pooling} : $p_{i,j} = \frac{1}{k^2}\sum_{(m,n)} x_{i+m, j+n}$
\end{itemize}

Avantages : invariance locale par translation, réduction dimensionnalité, augmentation champ réceptif.

\subsubsection{Architectures CNN emblématiques}

\textbf{LeNet-5 (1998)} :
\begin{verbatim}
Input (32×32) → Conv(6) → Pool → Conv(16) → Pool → FC(120) → FC(84) → Output(10)
\end{verbatim}

\textbf{AlexNet (2012)} :
8 couches, 60M paramètres, ReLU, Dropout, Data augmentation

\textbf{VGG (2014)} :
Empile petits filtres 3×3 (plus efficace que grands filtres)

\textbf{ResNet (2015)} :
Skip connections : $\mathbf{h}^{(l+2)} = \mathbf{h}^{(l)} + F(\mathbf{h}^{(l)})$

\subsubsection{Extension 3D}

Les CNN 3D remplacent convolutions 2D par 3D :
\[
(\mathbf{I} * \mathbf{K})_{i,j,k} = \sum_{l,m,n} \mathbf{I}_{i+l, j+m, k+n} \cdot \mathbf{K}_{l,m,n}
\]

\textbf{Coût computationnel} : Pour filtre $k \times k \times k$ :
\begin{itemize}
    \item 2D : $\mathcal{O}(k^2 HW)$ opérations
    \item 3D : $\mathcal{O}(k^3 HWD)$ opérations
\end{itemize}

Explosion cubique explique les contraintes mémoire des CNN 3D.

\subsection{Réseaux de neurones récurrents (RNN, LSTM, GRU)}

\subsubsection{RNN basiques}

Pour une séquence $(x_1, \ldots, x_T)$, un RNN maintient un état caché $h_t$ :
\[
h_t = \sigma(W_{hh} h_{t-1} + W_{xh} x_t + b_h)
\]

\textbf{Problème} : Vanishing/exploding gradients sur longues séquences.

\subsubsection{LSTM (Long Short-Term Memory)}

Introduit des gates (oubli, entrée, sortie) pour contrôler flux d'information :
\begin{align*}
f_t &= \sigma(W_f [h_{t-1}, x_t] + b_f) \quad \text{(forget gate)} \\
i_t &= \sigma(W_i [h_{t-1}, x_t] + b_i) \quad \text{(input gate)} \\
\tilde{C}_t &= \tanh(W_C [h_{t-1}, x_t] + b_C) \quad \text{(candidate)} \\
C_t &= f_t \odot C_{t-1} + i_t \odot \tilde{C}_t \quad \text{(cell state)} \\
o_t &= \sigma(W_o [h_{t-1}, x_t] + b_o) \quad \text{(output gate)} \\
h_t &= o_t \odot \tanh(C_t)
\end{align*}

\subsection{Transformers et mécanisme d'attention}

\subsubsection{Self-attention}

L'attention calcule une combinaison pondérée de valeurs :
\[
\text{Attention}(Q, K, V) = \text{softmax}\left(\frac{QK^T}{\sqrt{d_k}}\right) V
\]

où $Q$ (queries), $K$ (keys), $V$ (values) sont des projections linéaires de l'input.

\subsubsection{Multi-head attention}

Plusieurs têtes d'attention en parallèle capturent différents types de relations :
\[
\text{MultiHead}(Q, K, V) = \text{Concat}(\text{head}_1, \ldots, \text{head}_h) W^O
\]

\subsubsection{Architecture Transformer}

Un bloc Transformer combine :
\begin{enumerate}
    \item Multi-head self-attention
    \item Layer normalization + skip connection
    \item Feed-forward MLP
    \item Layer normalization + skip connection
\end{enumerate}

\textbf{Avantages} : Parallélisable, capture dépendances longue distance, scalable.

\textbf{Limitations} : Complexité quadratique $\mathcal{O}(n^2)$ en longueur de séquence.

\section{Techniques d'optimisation}

\subsection{Descente de gradient stochastique (SGD)}

\subsubsection{Gradient descent classique}

Pour minimiser $\mathcal{L}(\theta)$, on itère :
\[
\theta_{t+1} = \theta_t - \eta \nabla_\theta \mathcal{L}(\theta_t)
\]

où $\eta$ est le learning rate.

\textbf{Limitations} : Lent sur grandes données (calcul gradient sur dataset entier).

\subsubsection{Mini-batch SGD}

À chaque itération, utiliser un mini-batch $\mathcal{B}$ :
\[
\theta_{t+1} = \theta_t - \eta \nabla_\theta \frac{1}{|\mathcal{B}|} \sum_{i \in \mathcal{B}} \mathcal{L}_i(\theta_t)
\]

\textbf{Avantages} : Rapide, stochastique aide à échapper aux minima locaux, parallélisable sur GPU.

\subsection{Momentum et variantes}

\subsubsection{SGD avec momentum}

Accumule un vecteur de vélocité :
\begin{align*}
v_{t+1} &= \beta v_t + \nabla_\theta \mathcal{L}(\theta_t) \\
\theta_{t+1} &= \theta_t - \eta v_{t+1}
\end{align*}

Typiquement $\beta = 0.9$. Accélère dans directions constantes, amortit oscillations.

\subsubsection{Nesterov Accelerated Gradient (NAG)}

"Regarde avant de sauter" :
\[
v_{t+1} = \beta v_t + \nabla_\theta \mathcal{L}(\theta_t - \eta \beta v_t)
\]

\subsection{Optimiseurs adaptatifs}

\subsubsection{Adam (Adaptive Moment Estimation)}

Combine momentum et adaptation du learning rate par paramètre :
\begin{align*}
m_t &= \beta_1 m_{t-1} + (1-\beta_1) g_t \quad \text{(first moment)} \\
v_t &= \beta_2 v_{t-1} + (1-\beta_2) g_t^2 \quad \text{(second moment)} \\
\hat{m}_t &= \frac{m_t}{1 - \beta_1^t} \quad \text{(bias correction)} \\
\hat{v}_t &= \frac{v_t}{1 - \beta_2^t} \\
\theta_{t+1} &= \theta_t - \eta \frac{\hat{m}_t}{\sqrt{\hat{v}_t} + \epsilon}
\end{align*}

Hyperparamètres typiques : $\beta_1 = 0.9$, $\beta_2 = 0.999$, $\epsilon = 10^{-8}$.

\subsubsection{AdamW}

Variante corrigeant le weight decay :
\[
\theta_{t+1} = \theta_t - \eta \frac{\hat{m}_t}{\sqrt{\hat{v}_t} + \epsilon} - \eta \lambda \theta_t
\]

Préférable à Adam pour deep learning moderne.

\subsection{Learning rate scheduling}

\subsubsection{Step decay}

Réduire LR par facteur fixe tous les $N$ epochs :
\[
\eta_t = \eta_0 \cdot \gamma^{\lfloor t/N \rfloor}
\]

\subsubsection{Cosine annealing}

Variation douce :
\[
\eta_t = \eta_{\min} + \frac{1}{2}(\eta_{\max} - \eta_{\min})\left(1 + \cos\left(\frac{t}{T}\pi\right)\right)
\]

\subsubsection{ReduceLROnPlateau}

Adaptatif : réduit LR si métrique de validation stagne (notre choix pour GNN).

\subsection{Initialisation des poids}

\subsubsection{Xavier/Glorot initialization}

Pour activation linéaire/tanh :
\[
W \sim \mathcal{U}\left(-\sqrt{\frac{6}{n_{\text{in}} + n_{\text{out}}}}, \sqrt{\frac{6}{n_{\text{in}} + n_{\text{out}}}}\right)
\]

\subsubsection{He initialization}

Pour ReLU :
\[
W \sim \mathcal{N}\left(0, \frac{2}{n_{\text{in}}}\right)
\]

Compense le fait que ReLU annule 50\% des activations.

\subsection{Batch normalization et variantes}

\subsubsection{Batch Normalization}

Normalise activations par batch~\cite{Ioffe2015} :
\[
\hat{x} = \frac{x - \mu_\mathcal{B}}{\sqrt{\sigma_\mathcal{B}^2 + \epsilon}}
\]

puis applique transformation affine apprise : $y = \gamma \hat{x} + \beta$.

\textbf{Avantages} : Stabilise entraînement, permet learning rates plus élevés, régularise.

\subsubsection{Layer Normalization}

Normalise par features plutôt que par batch~\cite{Ba2016}. Préféré pour GNN (batch size petit, graphes de tailles variables).

\subsubsection{Graph Normalization}

Adaptations spécifiques pour graphes : GraphNorm, MeanSubtractionNorm.

\section{Régularisation}

Le sur-apprentissage (overfitting) survient quand le modèle mémorise les données d'entraînement sans généraliser. Plusieurs techniques combattent ce phénomène.

\subsection{Dropout}

\subsubsection{Principe}

Pendant l'entraînement, chaque neurone est désactivé avec probabilité $p$ (typiquement 0.5). À l'inférence, tous les neurones sont actifs mais leurs sorties sont multipliées par $(1-p)$.

\textbf{Formellement}, pour une couche :
\[
\mathbf{h}^{(l+1)} = \sigma\left(\mathbf{W}^{(l)} (\mathbf{h}^{(l)} \odot \mathbf{m}) + \mathbf{b}^{(l)}\right)
\]

où $\mathbf{m} \in \{0, 1\}^{d}$ est un masque aléatoire avec $P(m_i = 1) = 1-p$.

\textbf{Interprétation} : Dropout entraîne un ensemble de réseaux qui partagent des poids, équivalent à model averaging.

\subsubsection{Variantes}

\textbf{DropConnect} : Désactive connections plutôt que neurones.

\textbf{Spatial Dropout} : Pour CNN, désactive canaux entiers (cohérence spatiale).

\textbf{DropEdge} : Pour GNN, désactive arêtes aléatoirement (notre approche pour robustesse).

\subsection{Weight decay (L2 regularization)}

Ajoute pénalité sur magnitude des poids :
\[
\mathcal{L}_{\text{total}} = \mathcal{L}_{\text{data}} + \lambda \sum_l \|\mathbf{W}^{(l)}\|_F^2
\]

Typiquement $\lambda = 10^{-4}$ à $10^{-5}$.

\textbf{Effet} : Encourage poids petits, évite solutions extrêmes, améliore généralisation.

\subsection{Data augmentation}

\subsubsection{Augmentations géométriques}

Pour images~\cite{Shorten2019} :
\begin{itemize}
    \item Rotations aléatoires ($\pm 30°$)
    \item Translations ($\pm 10\%$)
    \item Scaling ($\times 0.8$ à $\times 1.2$)
    \item Flips horizontaux/verticaux
    \item Déformations élastiques
\end{itemize}

\subsubsection{Augmentations photométriques}

\begin{itemize}
    \item Variations de luminosité ($\pm 20\%$)
    \item Variations de contraste
    \item Ajout de bruit gaussien
    \item Flou gaussien
\end{itemize}

\subsubsection{Augmentations pour graphes}

Pour nos graphes cellulaires :
\begin{itemize}
    \item \textbf{Node dropout} : Retirer nœuds aléatoires
    \item \textbf{Edge dropout} : Retirer arêtes aléatoires
    \item \textbf{Feature masking} : Masquer certaines features
    \item \textbf{Subgraph sampling} : Échantillonner sous-graphes
    \item \textbf{Geometric transformations} : Rotations 3D aléatoires (test d'invariance)
\end{itemize}

\subsection{Early stopping}

\subsubsection{Principe}

Monitorer performance sur validation set. Arrêter si pas d'amélioration pendant $N$ epochs (patience).

\textbf{Algorithme} :
\begin{enumerate}
    \item Initialiser $\text{best\_val} = -\infty$, $\text{patience\_counter} = 0$
    \item À chaque epoch : calculer $\text{val\_score}$
    \item Si $\text{val\_score} > \text{best\_val}$ :
        \begin{itemize}
            \item Sauvegarder modèle
            \item $\text{best\_val} = \text{val\_score}$
            \item $\text{patience\_counter} = 0$
        \end{itemize}
    \item Sinon : $\text{patience\_counter} += 1$
    \item Si $\text{patience\_counter} \geq \text{patience}$ : arrêter
\end{enumerate}

\textbf{Notre configuration} : Patience de 20-30 epochs pour GNN.

\subsection{Label smoothing}

Remplace hard labels $(0, 1)$ par soft labels $(0.1, 0.9)$ :
\[
y_{\text{smooth}} = (1 - \epsilon) y_{\text{hard}} + \epsilon / K
\]

où $K$ est le nombre de classes, $\epsilon \approx 0.1$.

\textbf{Effet} : Réduit overconfidence, améliore calibration des probabilités.

\section{Bonnes pratiques d'entraînement}

\subsection{Validation croisée}

\subsubsection{K-fold cross-validation}

Diviser dataset en $K$ folds, entraîner $K$ fois en utilisant chaque fold comme validation. Moyenner les performances.

\textbf{Avantages} : Estimation robuste, utilise toutes les données.

\textbf{Inconvénient} : Coût computationnel $\times K$.

\textbf{Notre pratique} : 5-fold CV pour expériences finales.

\subsubsection{Stratified split}

Pour données déséquilibrées, assurer même distribution de classes dans train/val/test.

\subsection{Monitoring et visualisation}

\subsubsection{TensorBoard}

Monitoring en temps réel :
\begin{itemize}
    \item Courbes de loss (train/val)
    \item Métriques (accuracy, F1, AUC)
    \item Distributions de poids et gradients
    \item Histogrammes d'activations
\end{itemize}

\subsubsection{Weights \& Biases (Wandb)}

Plateforme cloud pour tracking d'expériences, comparaisons, partage de résultats.

\subsection{Checkpointing}

Sauvegarder régulièrement :
\begin{itemize}
    \item \textbf{Best model} : Meilleure performance validation
    \item \textbf{Latest model} : Dernier epoch
    \item \textbf{Epoch checkpoints} : Tous les N epochs (pour analyse post-hoc)
\end{itemize}

\textbf{Format} : Dictionnaire PyTorch incluant :
\begin{verbatim}
{
    'epoch': epoch,
    'model_state_dict': model.state_dict(),
    'optimizer_state_dict': optimizer.state_dict(),
    'scheduler_state_dict': scheduler.state_dict(),
    'train_loss': train_loss,
    'val_loss': val_loss,
    'val_metrics': metrics_dict,
    'config': config,
}
\end{verbatim}

\subsection{Hyperparamètre tuning}

\subsubsection{Grid search}

Test exhaustif de combinaisons. Coûteux mais complet.

\subsubsection{Random search}

Plus efficace que grid search pour espaces haute dimension.

\subsubsection{Bayesian optimization}

Modélise $f(\text{hyperparams}) \rightarrow \text{performance}$ par processus gaussien, propose séquentiellement les hyperparamètres à tester. Outils : Optuna, Hyperopt.

\subsubsection{Notre approche}

\begin{enumerate}
    \item Grid search grossier pour identifier régions prometteuses
    \item Random search fine autour de ces régions
    \item Validation finale avec 5-fold CV sur meilleure config
\end{enumerate}

\chapter{Compléments sur les graphes et GNNs}

Ce chapitre approfondit les aspects avancés de la théorie des graphes et des GNN non couverts dans le Chapitre 3.

\section{Expressivité et limitations théoriques des GNNs}

Cette section développe les aspects théoriques de l'expressivité des Graph Neural Networks, leurs limitations fondamentales, et les solutions proposées.

\subsection{Weisfeiler-Lehman test}

\subsubsection{Algorithme WL}

Le test de Weisfeiler-Lehman (1-WL) est un algorithme itératif testant l'isomorphisme de graphes :

\begin{enumerate}
    \item Initialiser chaque nœud avec un label (degré ou label initial)
    \item Itérer : nouveau label = hash(label ancien, multiset labels voisins)
    \item Comparer les multisets de labels finaux des deux graphes
\end{enumerate}

Si les multisets diffèrent, les graphes sont non-isomorphes. Sinon, inconnu.

\subsubsection{Lien avec les GNNs}

Xu et al.~\cite{Xu2019} ont prouvé que les GNNs utilisant somme+agrégation sont au plus aussi puissants que 1-WL. GIN atteint cette borne. GCN et GAT sont strictement moins expressifs.

\textbf{Implications :}
\begin{itemize}
    \item Certains graphes non-isomorphes sont indistinguables par GNNs standards
    \item Pour la classification d'organoïdes, c'est généralement suffisant (graphes très différents)
    \item Pour tâches nécessitant discrimination fine, des architectures plus puissantes (k-WL, higher-order) peuvent être nécessaires
\end{itemize}

Des analyses théoriques récentes~\cite{Morris2023WL,Grohe2023,Boker2023} ont approfondi notre compréhension de l'expressivité des GNNs, établissant des liens avec la logique et la complexité descriptive. Des extensions au-delà de 1-WL incluent les réseaux cellulaires~\cite{Bodnar2022CW} et simpliciaux~\cite{Bodnar2021Simplicial}, qui augmentent l'expressivité en opérant sur des structures topologiques plus riches que les graphes.

\subsection{Over-smoothing}

\subsubsection{Phénomène}

Avec un nombre de couches $K$ croissant, les représentations de tous les nœuds convergent vers une valeur commune, perdant toute information de structure.

\textbf{Intuition :}
Chaque couche "dilue" l'information en moyennant avec les voisins. Après $K$ couches, tous les nœuds ont accès à une information similaire (agrégation sur $K$-hop voisinage), rendant leurs représentations indistinguables.

\textbf{Analyse théorique :}
Pour GCN, les représentations convergent vers l'espace propre dominant du Laplacien normalisé.

\subsubsection{Conséquences pratiques}

\begin{itemize}
    \item Les GNNs standards sont limités à 2-4 couches en pratique
    \item Les réseaux profonds (> 8 couches) voient leurs performances s'effondrer
    \item Problématique pour graphes de grand diamètre nécessitant propagation longue distance
\end{itemize}

\subsubsection{Solutions proposées}

\begin{itemize}
    \item \textbf{Skip connections} : Connexions résiduelles préservant information initiale
    \item \textbf{Initial residual connections} : $\mathbf{h}_i^{(k+1)} = \mathbf{h}_i^{(k)} + \Delta\mathbf{h}_i^{(k)}$
    \item \textbf{Jumping Knowledge} : Concaténer représentations de toutes les couches
    \item \textbf{PairNorm, DGN} : Normalisation des représentations pour préserver diversité
\end{itemize}

\subsection{Over-squashing}

\subsubsection{Problème de goulets d'étranglement}

L'information doit traverser des goulets d'étranglement topologiques (nœuds de faible degré, coupes minimales) pour se propager. Cela cause une compression/perte d'information~\cite{Alon2021}, phénomène désormais bien compris théoriquement~\cite{DiGiovanni2023,Black2023}.

\textbf{Exemple :}
Dans un graphe en "haltère" (deux clusters denses connectés par un nœud pont), l'information des deux clusters doit passer par le nœud pont, créant un bottleneck.

\subsubsection{Relation avec courbure}

Des travaux récents~\cite{Topping2022,Nguyen2023} lient l'over-squashing à la courbure discrète des graphes (courbure d'Ollivier-Ricci). Les graphes de courbure négative (expansifs) facilitent la propagation ; les graphes de courbure positive (contractifs) l'entravent. Cette perspective géométrique offre des outils quantitatifs pour analyser et atténuer l'over-squashing.

\subsubsection{Atténuation}

\begin{itemize}
    \item Rewiring du graphe pour améliorer la connectivité
    \item Ajout d'arêtes longue-distance
    \item Architectures avec agrégation multi-échelles
\end{itemize}

\section{Types de graphes exotiques}

\subsection{Hypergraphes}

\subsubsection{Définition}

Un hypergraphe $\mathcal{H} = (V, \mathcal{E})$ généralise les graphes en permettant aux hyperarêtes $e \in \mathcal{E}$ de connecter un nombre arbitraire de nœuds : $e \subseteq V$ avec $|e| \geq 2$.

\textbf{Exemple biologique} : Une interaction protéique peut impliquer 3+ protéines simultanément.

\subsubsection{Hypergraph Neural Networks}

Extension des GNN aux hypergraphes via :
\begin{itemize}
    \item Message passing nœuds → hyperarêtes → nœuds
    \item Incidence matrix $\mathbf{H} \in \{0,1\}^{|V| \times |\mathcal{E}|}$
    \item Convolutions spectrales sur hypergraphes
\end{itemize}

\textbf{Application potentielle} : Modéliser interactions cellulaires multi-voies (signalisation paracrine, contacts mécaniques, gap junctions).

\subsection{Graphes dynamiques}

\subsubsection{Graphes temporels}

Un graphe temporel est une séquence $(G_1, G_2, \ldots, G_T)$ où la topologie et/ou les features évoluent.

\textbf{Cas d'usage pour organoïdes} : Tracking de croissance, division cellulaire, réorganisation spatiale.

\subsubsection{Temporal GNN}

Architectures combinant GNN spatial + RNN/LSTM temporel :
\[
\mathbf{h}_i^{(t)} = \text{GNN}\left(\{\mathbf{h}_j^{(t-1)} : j \in \mathcal{N}_i^{(t)}\}\right)
\]

\textbf{Applications} : Prédiction de dynamiques, détection d'événements (division, mort cellulaire).

\subsection{Graphes hétérogènes}

\subsubsection{Définition}

Un graphe hétérogène possède plusieurs types de nœuds $\mathcal{V} = \bigcup_{\tau \in \mathcal{T}_v} V_\tau$ et d'arêtes $\mathcal{E} = \bigcup_{\phi \in \mathcal{T}_e} E_\phi$.

\textbf{Exemple organoïdes} :
\begin{itemize}
    \item Types de nœuds : cellules souches, cellules différenciées, cellules en apoptose
    \item Types d'arêtes : contact direct, interaction paracrine, mechanical stress
\end{itemize}

\subsubsection{Heterogeneous GNN (HAN, RGCN)}

Message passing spécialisé par type :
\[
\mathbf{h}_i^{(l+1)} = \sigma\left(\sum_{\phi \in \mathcal{T}_e} \sum_{j \in \mathcal{N}_i^\phi} \mathbf{W}_\phi^{(l)} \mathbf{h}_j^{(l)}\right)
\]

\section{Geometric Deep Learning : formalisme unifié}

\subsection{Principe fondamental}

Le Geometric Deep Learning~\cite{Bronstein2021} propose un cadre théorique unifié pour le deep learning sur domaines non-euclidiens (graphes, variétés, groupes, ensembles) basé sur les symétries et invariances.

\subsubsection{Théorème fondamental}

Toute architecture de deep learning peut être caractérisée par :
\begin{enumerate}
    \item Le \textbf{domaine} sur lequel elle opère (grille, graphe, variété, etc.)
    \item Les \textbf{symétries} qu'elle respecte (translations, rotations, permutations, etc.)
    \item L'\textbf{échelle} à laquelle elle opère (locale, globale, multi-échelle)
\end{enumerate}

\subsection{Hiérarchie des symétries}

\textbf{Grilles régulières (images)} :
\begin{itemize}
    \item Symétrie : groupe de translation $\mathbb{Z}^d$
    \item Architecture : CNN (convolution = équivariance par translation)
\end{itemize}

\textbf{Ensembles (point clouds)} :
\begin{itemize}
    \item Symétrie : permutations $S_n$
    \item Architecture : DeepSets~\cite{Zaheer2017} (théorème de représentation universelle pour fonctions invariantes aux permutations : $f(\mathcal{X}) = \rho(\sum_i \phi(x_i))$), PointNet~\cite{Qi2017} (application aux nuages de points 3D avec max-pooling)
    \item Limitation : agrégation globale, pas de structure locale
\end{itemize}

\textbf{Graphes} :
\begin{itemize}
    \item Symétrie : permutations des nœuds
    \item Architecture : GNN (message passing invariant)
\end{itemize}

\textbf{Graphes géométriques} :
\begin{itemize}
    \item Symétrie : groupe euclidien $E(d)$ (translations + rotations + réflexions)
    \item Architecture : EGNN, SchNet (équivariance géométrique)
\end{itemize}

\subsection{Blueprint du Geometric Deep Learning}

Recette générale pour concevoir une architecture :
\begin{enumerate}
    \item Identifier le domaine et ses symétries
    \item Construire des opérations linéaires équivariantes
    \item Ajouter des non-linéarités point-wise (préservent équivariance)
    \item Composer en réseau profond
\end{enumerate}

\section{Topological Data Analysis et graphes}

\subsection{Persistent homology}

\subsubsection{Principe}

La persistent homology caractérise la topologie de données via naissance/mort de features topologiques (composantes connexes, trous, cavités) à différentes échelles.

\textbf{Filtration} : Construire séquence de complexes simpliciaux $K_0 \subseteq K_1 \subseteq \cdots \subseteq K_n$ en augmentant un paramètre (ex: rayon).

\textbf{Homologie} : Compter composantes connexes ($H_0$), trous ($H_1$), cavités ($H_2$).

\subsubsection{Diagrammes de persistance}

Visualisation : chaque feature $(b, d)$ où $b$ = naissance, $d$ = mort.

Features persistantes (longue durée de vie) = structures topologiques significatives.

\subsection{Applications aux graphes biologiques}

\textbf{Organoïdes} : 
\begin{itemize}
    \item $H_0$ : Nombre de clusters cellulaires
    \item $H_1$ : Présence de lumens (cavités)
    \item $H_2$ : Structures 3D complexes
\end{itemize}

\textbf{Perspective} : Enrichir features de graphes avec descripteurs topologiques pour améliorer classification.

\section{Expressivité théorique : au-delà du WL-test}

\subsection{Rappel : limitations du 1-WL test}

Le 1-WL test (Weisfeiler-Leman) itère :
\begin{align*}
c_i^{(0)} &= \text{label initial} \\
c_i^{(k+1)} &= \text{HASH}\left(c_i^{(k)}, \{\!\!\{c_j^{(k)} : j \in \mathcal{N}(i)\}\!\!\}\right)
\end{align*}

La plupart des GNN standards (GCN, GAT, GraphSAGE) sont au mieux aussi puissants que 1-WL~\cite{Xu2019}.

\subsection{Hiérarchie WL}

\subsubsection{k-WL test}

Le k-WL opère sur k-tuples de nœuds au lieu de nœuds individuels.

\textbf{2-WL} : Considère paires $(i, j)$, peut distinguer plus de graphes.

\textbf{k-WL pour $k \geq 3$} : Encore plus puissant, mais coût combinatoire $\mathcal{O}(n^k)$.

\subsubsection{Higher-order GNNs}

\textbf{k-GNN}~\cite{Morris2019} : Opère sur k-tuples, atteint expressivité k-WL.

\textbf{Coût} : $\mathcal{O}(n^k)$ nœuds dans le graphe augmenté. Prohibitif pour $k \geq 3$.

\subsection{Alternatives à higher-order}

\textbf{Subgraph GNN} : Compte occurrences de motifs (triangles, cycles).

\textbf{Random features} : Ajoute features aléatoires pour briser symétries.

\textbf{Positional encodings} : Encode position relative des nœuds.

\subsection{En pratique : expressivité nécessaire ?}

Pour la plupart des tâches réelles, 1-WL expressivité suffit. Les graphes pathologiques non-distinguables par 1-WL sont rares en pratique.

\textbf{Notre contexte (organoïdes)} : Graphes géométriques avec positions 3D fournissent déjà information suffisante pour distinguer structures. Pas besoin de higher-order.

\section{Message passing généralisé et extensions}

\subsection{Message passing avec edge updates}

Au-delà du MPNN standard, mettre à jour aussi les arêtes :
\begin{align*}
\mathbf{m}_{ij} &= \phi_e\left(\mathbf{h}_i, \mathbf{h}_j, \mathbf{e}_{ij}\right) \\
\mathbf{e}_{ij}' &= \mathbf{e}_{ij} + \mathbf{m}_{ij} \\
\mathbf{h}_i' &= \phi_v\left(\mathbf{h}_i, \sum_{j \in \mathcal{N}(i)} \mathbf{m}_{ij}\right)
\end{align*}

\textbf{Utilité} : Raffiner représentation des relations au fil des couches.

\subsection{Attention multi-échelles}

Combiner attention locale (voisins directs) et globale (tous les nœuds) :
\[
\mathbf{h}_i' = \alpha_{\text{local}} \sum_{j \in \mathcal{N}(i)} a_{ij} \mathbf{h}_j + \alpha_{\text{global}} \sum_{j \in V} a_{ij}' \mathbf{h}_j
\]

\textbf{Graph Transformers} : Attention sur tous les nœuds (global), complexité $\mathcal{O}(n^2)$.

\subsection{Graph pooling hierarchical}

\subsubsection{Motivation}

Réduction progressive de la taille du graphe pour capture multi-échelle.

\subsubsection{Méthodes}

\textbf{Top-K pooling} : Garde top-K nœuds selon un score appris.

\textbf{DiffPool}~\cite{Ying2018} : Assigne nœuds à clusters de manière différentiable :
\[
\mathbf{S} = \text{softmax}(\text{GNN}(\mathbf{X}, \mathbf{A}))
\]

Nouveau graphe coarse : $\mathbf{X}' = \mathbf{S}^T \mathbf{X}$, $\mathbf{A}' = \mathbf{S}^T \mathbf{A} \mathbf{S}$.

\textbf{SAGPool, EdgePool, etc.} : Variantes avec différentes stratégies.

\subsubsection{Application organoïdes}

Hierarchical pooling pourrait capturer :
\begin{itemize}
    \item Niveau 1 : Cellules individuelles
    \item Niveau 2 : Micro-domaines (clusters de cellules similaires)
    \item Niveau 3 : Régions fonctionnelles (cryptes, lumens)
    \item Niveau 4 : Organoïde entier
\end{itemize}

\textbf{Perspective future} : À explorer pour classification fine.

\chapter{Théorie des processus ponctuels}

Ce chapitre fournit les fondements mathématiques des processus ponctuels spatiaux utilisés pour notre génération de données synthétiques. Pour des traitements complets, voir~\cite{Illian2008,Diggle2013,Baddeley2015}.

\section{Processus de Poisson : propriétés et simulation}

\subsection{Définition rigoureuse}

\subsubsection{Processus ponctuel}

Un processus ponctuel $\Phi$ sur un espace $\mathcal{X}$ est une variable aléatoire à valeurs dans l'ensemble des configurations de points $\mathcal{N} = \{n \subseteq \mathcal{X} : n \text{ localement fini}\}$.

Pour un domaine borné $D \subset \mathbb{R}^d$, $\Phi(D)$ est le nombre de points dans $D$.

\subsubsection{Processus de Poisson homogène}

Un processus de Poisson sur domaine $D$ d'intensité $\lambda > 0$ satisfait :

\textbf{P1. Nombre de points} : $N(D) = \Phi(D) \sim \text{Poisson}(\lambda |D|)$ où $|D|$ est le volume de $D$.

\textbf{P2. Indépendance spatiale} : Pour régions disjointes $B_1, \ldots, B_k$, les $N(B_i)$ sont indépendants.

\textbf{P3. Uniformité} : Conditionnellement à $N(D) = n$, les $n$ points sont i.i.d. uniformes dans $D$.

\subsubsection{Fonction d'intensité}

L'intensité $\lambda$ représente le nombre moyen de points par unité de volume :
\[
\mathbb{E}[N(B)] = \lambda |B|
\]

\subsection{Propriétés mathématiques}

\subsubsection{Superposition}

Si $\Phi_1, \ldots, \Phi_k$ sont des processus de Poisson indépendants d'intensités $\lambda_1, \ldots, \lambda_k$, alors leur superposition $\Phi = \bigcup_i \Phi_i$ est un processus de Poisson d'intensité $\sum_i \lambda_i$.

\subsubsection{Thinning}

Si on garde chaque point d'un processus de Poisson $\Phi(\lambda)$ indépendamment avec probabilité $p$, le résultat est un processus de Poisson $\Phi(p\lambda)$.

\textbf{Utilité} : Génération de processus inhomogènes par rejection sampling.

\subsubsection{Campbell's theorem}

Pour une fonction $f : \mathcal{X} \rightarrow \mathbb{R}$ :
\[
\mathbb{E}\left[\sum_{x \in \Phi} f(x)\right] = \int_\mathcal{X} f(x) \lambda(x) dx
\]

\subsection{Simulation}

\subsubsection{Algorithme basique (domaine borné)}

Pour un processus homogène d'intensité $\lambda$ sur domaine $D$ :
\begin{enumerate}
    \item Tirer $N \sim \text{Poisson}(\lambda |D|)$
    \item Tirer $N$ points i.i.d. uniformément dans $D$
\end{enumerate}

\subsubsection{Simulation dans une sphère 3D}

Pour une sphère de rayon $R$ :

\textbf{Méthode 1 : Rejection sampling}
\begin{enumerate}
    \item Tirer $(x, y, z) \sim \mathcal{U}([-R, R]^3)$
    \item Accepter si $x^2 + y^2 + z^2 \leq R^2$
    \item Répéter jusqu'à obtenir $N$ points
\end{enumerate}

\textbf{Méthode 2 : Coordonnées sphériques}
\begin{enumerate}
    \item Tirer $r \sim$ avec densité $p(r) \propto r^2$ sur $[0, R]$ (correction volumétrique)
    \item Tirer $\theta \sim \mathcal{U}([0, 2\pi])$ (azimuth)
    \item Tirer $\cos\phi \sim \mathcal{U}([-1, 1])$ (polar, uniforme sur sphère)
    \item Convertir : $x = r\sin\phi\cos\theta$, $y = r\sin\phi\sin\theta$, $z = r\cos\phi$
\end{enumerate}

\textbf{Notre implémentation} : Méthode 1 (rejection) par simplicité.

\section{Processus de Poisson inhomogènes}

\subsection{Définition}

Un processus de Poisson inhomogène possède une fonction d'intensité spatiale $\lambda(\mathbf{x})$ variant dans l'espace :
\[
\mathbb{E}[N(B)] = \int_B \lambda(\mathbf{x}) d\mathbf{x}
\]

\textbf{Interprétation biologique} : Modélise gradients de densité cellulaire (centre vs périphérie).

\subsection{Simulation par thinning}

\begin{enumerate}
    \item Générer processus de Poisson homogène d'intensité $\lambda_{\max} = \sup_{\mathbf{x}} \lambda(\mathbf{x})$
    \item Pour chaque point $\mathbf{x}_i$, le garder avec probabilité $p(\mathbf{x}_i) = \lambda(\mathbf{x}_i) / \lambda_{\max}$
\end{enumerate}

\subsection{Exemples de fonctions d'intensité}

\subsubsection{Gradient radial}

\[
\lambda(x, y, z) = \lambda_{\max} \exp\left(-\alpha \frac{\|(x,y,z)\|}{R}\right)
\]

Intensité décroît exponentiellement du centre à la périphérie.

\subsubsection{Gradient linéaire}

\[
\lambda(x, y, z) = \lambda_{\min} + (\lambda_{\max} - \lambda_{\min}) \frac{z + R}{2R}
\]

Gradient le long d'un axe (modélise polarisation).

\section{Processus de Cox et processus log-gaussiens}

\subsection{Processus de Cox (doubly stochastic)}

Un processus de Cox est un processus de Poisson dont l'intensité $\Lambda(\mathbf{x})$ est elle-même un champ aléatoire.

\textbf{Construction} :
\begin{enumerate}
    \item Tirer un champ d'intensité $\Lambda(\mathbf{x})$ d'une distribution (ex: log-normal)
    \item Conditionnellement à $\Lambda$, générer Poisson d'intensité $\Lambda(\mathbf{x})$
\end{enumerate}

\subsection{Processus log-gaussiens}

Cas particulier où $\log \Lambda(\mathbf{x})$ est un champ gaussien :
\[
\log \Lambda(\mathbf{x}) = \mu + Z(\mathbf{x})
\]

où $Z$ est un processus gaussien de moyenne 0 et covariance $C(\mathbf{x}, \mathbf{x}')$.

\textbf{Propriété} : La corrélation spatiale dans $\Lambda$ induit du clustering dans les points générés.

\textbf{Application} : Modéliser variabilité expérimentale entre organoïdes (même protocole, densités variables).

\section{Processus de Gibbs et modèles énergétiques}

\subsection{Formulation générale}

Un processus de Gibbs définit une distribution via une densité de probabilité :
\[
f(\mathbf{x}) = \frac{1}{Z} \exp(-U(\mathbf{x}))
\]

où $U(\mathbf{x})$ est l'énergie de la configuration, $Z$ la constante de normalisation (partition function).

\subsection{Processus de Matérn (clustering)}

\subsubsection{Construction hiérarchique}

Le processus de Matérn~\cite{Matern1960} génère clusters via :
\begin{enumerate}
    \item Générer centres de clusters via Poisson d'intensité $\kappa$ (parents)
    \item Autour de chaque parent, générer Poisson d'intensité $\mu$ dans rayon $r$ (offspring)
    \item Retirer les parents, garder seulement offspring
\end{enumerate}

\subsubsection{Paramètres et contrôle}

\begin{itemize}
    \item $\kappa$ : intensité de parents (contrôle nombre de clusters)
    \item $\mu$ : offspring par parent (contrôle taille clusters)
    \item $r$ : rayon de clustering (contrôle compacité)
\end{itemize}

\textbf{Interprétation biologique} : Cellules issues de même précurseur restent proches (lignage commun).

\subsubsection{Fonction K de Ripley attendue}

Pour Matérn, $K(t)$ exhibe valeurs supérieures à Poisson ($K_{\text{Poisson}}(t) = (4/3)\pi t^3$) pour $t < r$, indiquant clustering.

\subsection{Processus de Strauss (répulsion)}

\subsubsection{Fonction d'énergie}

Le processus de Strauss~\cite{Strauss1975} définit :
\[
U(\mathbf{x}) = -\alpha n(\mathbf{x}) - \beta s_R(\mathbf{x})
\]

où :
\begin{itemize}
    \item $n(\mathbf{x})$ : nombre de points
    \item $s_R(\mathbf{x})$ : nombre de paires de points à distance $< R$
    \item $\beta < 0$ : pénalité pour proximité (répulsion)
\end{itemize}

\subsubsection{Hard-core}

Cas extrême : $\beta = -\infty$, aucune paire à distance $< R$ (exclusion stricte).

\textbf{Interprétation} : Modélise exclusion stérique entre cellules (volume incompressible).

\subsubsection{Simulation}

Pas de simulation directe. Utilisation de MCMC (Metropolis-Hastings) :
\begin{enumerate}
    \item Initialiser avec Poisson
    \item Proposer modifications (ajouter/retirer/déplacer point)
    \item Accepter/rejeter selon ratio de Metropolis-Hastings
    \item Itérer jusqu'à convergence
\end{enumerate}

\textbf{Défis} : Convergence lente, tuning des propositions.

\textbf{Notre implémentation} : 10,000 itérations MCMC, taux d'acceptation ~30\%.

\section{Estimation statistique et inférence}

\subsection{Maximum de vraisemblance}

\subsubsection{Vraisemblance pour Poisson}

Pour processus de Poisson homogène, la log-vraisemblance est :
\[
\log L(\lambda ; \mathbf{x}) = n \log \lambda - \lambda |D| + \text{const}
\]

\textbf{MLE} : $\hat{\lambda} = n / |D|$ (estimateur naturel).

\subsubsection{Pour processus complexes}

Pour Gibbs/Strauss, la constante $Z$ est intractable (intégrale sur configurations). Méthodes alternatives nécessaires.

\subsection{Méthodes basées simulation}

\subsubsection{Approximate Bayesian Computation (ABC)}

Inférence sans vraisemblance explicite :
\begin{enumerate}
    \item Proposer paramètres $\theta$ du prior
    \item Simuler données $\mathbf{x}_{\text{sim}}$ avec ces paramètres
    \item Calculer distance $d(\mathbf{x}_{\text{sim}}, \mathbf{x}_{\text{obs}})$ (ex: via statistiques K, F, G)
    \item Accepter $\theta$ si $d < \epsilon$
\end{enumerate}

\textbf{Application} : Valider que nos paramètres Matérn/Strauss génèrent patterns compatibles avec données réelles.

\subsection{Pseudo-likelihood}

Pour Gibbs, remplace vraisemblance complète par produit de vraisemblances conditionnelles :
\[
PL(\theta) = \prod_i f(x_i \mid \mathbf{x}_{-i}; \theta)
\]

Plus tractable, permet estimation via optimisation.

\section{Processus ponctuels sur variétés}

\subsection{Extension aux sphères}

\subsubsection{Processus sur $\mathbb{S}^2$}

Pour la sphère 2D, adapter fonctions K, F, G en tenant compte de la géométrie sphérique.

\textbf{Distance géodésique} : Arc length sur sphère au lieu de distance euclidienne.

Pour deux points $\mathbf{p}_1, \mathbf{p}_2$ sur sphère de rayon $R$ :
\[
d_{\text{geo}}(\mathbf{p}_1, \mathbf{p}_2) = R \arccos\left(\frac{\mathbf{p}_1 \cdot \mathbf{p}_2}{R^2}\right)
\]

\subsubsection{Fonction K adaptée}

\[
K(r) = \frac{1}{\lambda} \mathbb{E}\left[\sum_{i \neq j} \mathbb{1}(d_{\text{geo}}(x_i, x_j) < r)\right]
\]

Pour Poisson sur sphère, $K_{\text{Poisson}}(r) = 4\pi R^2 (1 - \cos(r/R))$.

\subsection{Processus sur surfaces courbes générales}

\subsubsection{Variétés riemanniennes}

Généralisation aux variétés $\mathcal{M}$ avec métrique riemannienne $g$.

\textbf{Intensité} : $\lambda(\mathbf{x})$ densité par rapport à mesure de volume riemannienne.

\textbf{Applications futures} : 
\begin{itemize}
    \item Organoïdes non-sphériques (tubulaires, bourgeonnés)
    \item Surfaces d'organes réels (cortex cérébral plissé)
\end{itemize}

\section{Statistiques de second ordre}

\subsection{Fonction K de Ripley}

\subsubsection{Définition}

Pour un processus d'intensité $\lambda$ :
\[
K(r) = \frac{1}{\lambda} \mathbb{E}[\text{nombre de points à distance} < r \text{ d'un point typique}]
\]

\textbf{Estimateur empirique} (correction des effets de bord) :
\[
\hat{K}(r) = \frac{|D|}{n(n-1)} \sum_{i=1}^n \sum_{j \neq i} \mathbb{1}(d_{ij} < r) w_{ij}
\]

où $w_{ij}$ corrige les effets de bord.

\subsubsection{Fonction L (variance stabilisée)}

\[
L(r) = \sqrt{\frac{K(r)}{\pi}} - r \quad \text{(2D)}, \quad L(r) = \left(\frac{3K(r)}{4\pi}\right)^{1/3} - r \quad \text{(3D)}
\]

\textbf{Interprétation} :
\begin{itemize}
    \item $L(r) = 0$ : Poisson (random)
    \item $L(r) > 0$ : Clustering
    \item $L(r) < 0$ : Régularité/répulsion
\end{itemize}

\subsection{Fonction F (nearest neighbor)}

Distribution de la distance au plus proche voisin :
\[
F(r) = P(\text{distance au NN} \leq r)
\]

Pour Poisson 3D : $F_{\text{Poisson}}(r) = 1 - \exp\left(-\frac{4\pi\lambda r^3}{3}\right)$.

\subsection{Fonction G (event-to-event)}

Distribution de distance entre points :
\[
G(r) = P(\text{distance d'un point typique à son NN} \leq r)
\]

Pour Poisson : $G = F$ (propriété de Slivnyak).

\subsection{Test d'hypothèse CSR}

\subsubsection{Enveloppes de Monte Carlo}

Pour tester $H_0$ : "données issues de Poisson" :
\begin{enumerate}
    \item Simuler $M$ réalisations de Poisson (ex: $M=99$)
    \item Calculer $K(r)$ pour chaque simulation
    \item Construire enveloppes : min, max à chaque $r$
    \item Comparer $K_{\text{obs}}(r)$ aux enveloppes
\end{enumerate}

Si $K_{\text{obs}}(r)$ sort des enveloppes : rejet de $H_0$.

\subsubsection{Tests formels}

\textbf{Test de Kolmogorov-Smirnov} : Comparer distributions $F_{\text{obs}}$ vs $F_{\text{Poisson}}$.

\textbf{Test du $\chi^2$} : Sur histogramme des distances NN.

\section{Notre utilisation pour validation}

\subsection{Protocole de validation synthétiques}

Pour chaque type de processus généré :
\begin{enumerate}
    \item Calculer $\hat{K}(r)$, $\hat{L}(r)$ empiriques
    \item Comparer aux valeurs théoriques attendues
    \item Construire enveloppes Monte Carlo (99 simulations)
    \item Vérifier que données synthétiques tombent dans enveloppes
\end{enumerate}

\subsection{Résultats de validation}

[À compléter avec vos résultats - exemples de figures K/L pour chaque processus]

\textbf{Poisson} : $L(r) \approx 0$ pour tous $r$ (attendu).

\textbf{Matérn} : $L(r) > 0$ pour $r < 30$ μm, pic clustering à $r \approx 15$ μm.

\textbf{Strauss} : $L(r) < 0$ pour $r < 20$ μm, indiquant répulsion effective.

\section{Processus de Cox et processus log-gaussiens}

Les processus de Cox généralisent Poisson avec une intensité $\Lambda$ elle-même aléatoire. Les processus log-gaussiens modélisent $\log \Lambda$ par un champ gaussien, permettant d'incorporer de la corrélation spatiale.

\section{Processus de Gibbs et modèles énergétiques}

Les processus de Gibbs définissent une distribution via une énergie :
\[
P(\mathbf{x}) \propto \exp(-U(\mathbf{x}))
\]
Exemples : processus de Strauss (répulsion), processus de Matérn (clustering).

\section{Estimation statistique et inférence}

\subsection{Maximum de vraisemblance}
Difficile pour processus complexes, nécessite calcul de constante de normalisation.

\subsection{Méthodes basées simulation}
ABC (Approximate Bayesian Computation) permet l'inférence sans vraisemblance explicite.

\section{Processus ponctuels sur variétés}

\subsection{Extension aux sphères}
Pour un processus sur la sphère $\mathbb{S}^2$, les fonctions K, F, G sont adaptées en tenant compte de la géométrie sphérique (distances géodésiques).

\subsection{Processus sur surfaces courbes}
Généralisation aux variétés riemanniennes quelconques, pertinent pour modéliser des organes de formes complexes.

\chapter{Détails d'implémentation}

Ce chapitre décrit les aspects techniques de l'implémentation logicielle, permettant la reproduction complète de nos résultats.

\section{Technologies et bibliothèques}

\subsection{Environnement logiciel complet}

\subsubsection{Langage et frameworks}

\textbf{Python 3.9} : Langage principal pour sa richesse d'écosystème scientifique.

\textbf{PyTorch 2.0}~\cite{Paszke2019} : Framework de deep learning choisi pour :
\begin{itemize}
    \item API pythonique et intuitive
    \item Graphe computationnel dynamique (debug facile)
    \item Communauté active et documentation excellente
    \item Support GPU/TPU mature
    \item Intégration native avec PyTorch Geometric
\end{itemize}

\textbf{PyTorch Geometric 2.3}~\cite{Fey2019} : Bibliothèque spécialisée GNN :
\begin{itemize}
    \item Implémentations optimisées de GCN, GAT, GraphSAGE, etc.
    \item Data structures efficaces pour graphes
    \item Mini-batching intelligent
    \item CUDA kernels optimisés (scatter/gather operations)
\end{itemize}

\subsubsection{Traitement d'images}

\textbf{Cellpose 2.2}~\cite{Stringer2021} : Segmentation cellulaire state-of-the-art
\begin{itemize}
    \item Modèle cyto2 pré-entraîné
    \item Support 3D natif
    \item Fine-tuning possible sur nos données
\end{itemize}

\textbf{scikit-image 0.20}~\cite{VanDerWalt2014} : Opérations morphologiques
\begin{itemize}
    \item Filtrage (gaussien, médian, bilatéral)
    \item Transformations géométriques
    \item Extraction de features (regionprops 3D)
    \item Label manipulation
\end{itemize}

\textbf{OpenCV 4.7} : Opérations bas-niveau optimisées

\subsubsection{Calcul scientifique}

\textbf{NumPy 1.24} : Arrays N-dimensionnels, opérations vectorisées

\textbf{SciPy 1.10} : 
\begin{itemize}
    \item \texttt{scipy.spatial} : cKDTree (K-NN efficient), Delaunay, distance matrices
    \item \texttt{scipy.ndimage} : Filtres 3D, morphologie mathématique
    \item \texttt{scipy.stats} : Tests statistiques
\end{itemize}

\textbf{Pandas 2.0} : Manipulation de données tabulaires (métadonnées, résultats)

\subsubsection{Machine Learning classique}

\textbf{scikit-learn 1.2} :
\begin{itemize}
    \item Baselines (Random Forest, SVM)
    \item Métriques (accuracy, F1, confusion matrix)
    \item Preprocessing (StandardScaler, train\_test\_split)
    \item Model selection (GridSearchCV)
\end{itemize}

\subsection{Visualisation et monitoring}

\subsubsection{Visualisation statique}

\textbf{Matplotlib 3.7} : Plotting standard
\begin{itemize}
    \item Courbes d'apprentissage
    \item Matrices de confusion
    \item Distributions de features
\end{itemize}

\textbf{Seaborn 0.12} : Visualisations statistiques high-level
\begin{itemize}
    \item Heatmaps
    \item Pairplots
    \item Distribution plots
\end{itemize}

\subsubsection{Visualisation 3D interactive}

\textbf{PyVista 0.38} : Visualisation scientifique 3D
\begin{itemize}
    \item Rendering de graphes cellulaires
    \item Export HTML interactif
    \item Volume rendering d'images
\end{itemize}

\textbf{Plotly 5.13} : Visualisations web interactives
\begin{itemize}
    \item Scatter 3D interactifs
    \item Dashboard de résultats
\end{itemize}

\subsubsection{Monitoring d'entraînement}

\textbf{TensorBoard 2.12} : Visualisation temps-réel
\begin{itemize}
    \item Scalars (loss, metrics)
    \item Histogrammes (poids, gradients)
    \item Graphs (architecture)
    \item Embeddings (projections)
\end{itemize}

\textbf{Weights \& Biases (Wandb)} : Tracking cloud
\begin{itemize}
    \item Comparaison d'expériences
    \item Hyperparameter sweeps
    \item Collaboration en équipe
    \item Hosting de modèles
\end{itemize}

\subsection{Infrastructure et déploiement}

\textbf{Docker} : Containerisation pour reproductibilité
\begin{verbatim}
FROM pytorch/pytorch:2.0.0-cuda11.8-cudnn8-runtime
RUN pip install torch-geometric cellpose ...
\end{verbatim}

\textbf{Git + GitHub} : Versioning de code
\begin{itemize}
    \item Branches : main, develop, feature/xxx
    \item CI/CD : GitHub Actions (tests, linting)
    \item Releases : versions tagguées
\end{itemize}

\section{Architecture logicielle du pipeline}

\subsection{Organisation modulaire}

Notre codebase (~5,000 lignes) est organisée en 9 modules Python :

\begin{verbatim}
code/
├── data/                    # Dataset management
│   ├── dataset.py          # PyG Dataset classes
│   ├── loader.py           # DataLoaders
│   └── preprocessing.py    # Image preprocessing
├── models/                  # GNN architectures
│   ├── gcn.py              # 250 lignes
│   ├── gat.py              # 280 lignes
│   ├── graphsage.py        # 180 lignes
│   ├── gin.py              # 200 lignes
│   ├── egnn.py             # 320 lignes
│   └── classifier.py       # Interface unifiée
├── utils/                   # Utilities
│   ├── segmentation.py     # Cellpose wrapper (300 lignes)
│   ├── graph_builder.py    # Graph construction (350 lignes)
│   ├── features.py         # Feature extraction (300 lignes)
│   └── metrics.py          # Evaluation metrics
├── synthetic/               # Synthetic generation
│   ├── point_processes.py  # Spatial processes (450 lignes)
│   ├── generator.py        # Organoid generator (350 lignes)
│   └── statistics.py       # Ripley's K, etc.
├── scripts/                 # Execution scripts
│   ├── train.py            # Training (350 lignes)
│   ├── evaluate.py         # Evaluation (200 lignes)
│   └── generate_data.py    # Data generation
└── visualization/           # Visualization
    ├── plot_graphs.py      # 2D/3D plots
    ├── plot_3d.py          # Interactive viewer
    └── interpretability.py # GNNExplainer
\end{verbatim}

\subsection{Patterns de conception}

\subsubsection{Factory pattern}

Classe \texttt{OrganoidClassifier} unifie création de modèles :
\begin{verbatim}
model = OrganoidClassifier.create(
    model_type='gcn',  # or 'gat', 'gin', etc.
    in_channels=10,
    num_classes=5,
    hidden_channels=128,
)
\end{verbatim}

\subsubsection{Builder pattern}

\texttt{GraphBuilder} avec stratégies configurables :
\begin{verbatim}
builder = GraphBuilder(
    edge_method='knn',
    k_neighbors=8,
)
graph = builder.build_graph(masks, features, label)
\end{verbatim}

\subsubsection{Strategy pattern}

Différentes stratégies de pooling, agrégation, normalization interchangeables.

\subsection{Design pour extensibilité}

\textbf{Abstraction} : Classes de base abstraites pour Dataset, Model, Trainer

\textbf{Plugins} : Nouvelles architectures GNN ajoutables sans modifier code existant

\textbf{Configuration} : Hyperparamètres externes (YAML, pas hard-codés)

\section{Gestion des ressources computationnelles}

\subsection{Hardware utilisé}

\textbf{Développement et tests} :
\begin{itemize}
    \item GPU : NVIDIA RTX 3090 (24 GB VRAM)
    \item CPU : AMD Ryzen 9 5950X (16 cores)
    \item RAM : 64 GB DDR4
    \item Stockage : 2 TB NVMe SSD
\end{itemize}

\textbf{Entraînements finaux} :
\begin{itemize}
    \item GPU : NVIDIA A100 (40 GB VRAM) ou V100 (32 GB)
    \item Cluster : 4-8 GPUs en data-parallel
    \item Temps typique : 4-6h pour modèle complet
\end{itemize}

\subsection{Optimisations mémoire}

\subsubsection{Mixed precision training (FP16)}

Utilisation de \texttt{torch.cuda.amp} (Automatic Mixed Precision) :
\begin{verbatim}
scaler = torch.cuda.amp.GradScaler()
with torch.cuda.amp.autocast():
    output = model(data)
    loss = criterion(output, labels)
scaler.scale(loss).backward()
scaler.step(optimizer)
scaler.update()
\end{verbatim}

\textbf{Gains} :
\begin{itemize}
    \item Réduction mémoire : 2×
    \item Accélération calculs : 1.5-2× (Tensor Cores)
    \item Précision : aucune perte pour nos tâches
\end{itemize}

\subsubsection{Gradient accumulation}

Pour simuler large batch avec mémoire limitée :
\begin{verbatim}
accumulation_steps = 4
for i, batch in enumerate(loader):
    loss = model(batch) / accumulation_steps
    loss.backward()
    if (i + 1) % accumulation_steps == 0:
        optimizer.step()
        optimizer.zero_grad()
\end{verbatim}

\textbf{Équivalent} : batch size effectif = batch\_size × accumulation\_steps

\subsubsection{Gradient checkpointing}

Trade-off temps/mémoire : recalculer activations en backward au lieu de les stocker.

\textbf{Usage} : Modèles très profonds (6+ couches) sur graphes larges (1000+ nœuds).

\subsection{Optimisations vitesse}

\subsubsection{DataLoader multi-process}

\begin{verbatim}
loader = DataLoader(
    dataset,
    batch_size=32,
    num_workers=4,      # 4 processus parallèles
    pin_memory=True,    # Transfert CPU→GPU plus rapide
    persistent_workers=True,  # Réutilise workers
)
\end{verbatim}

\textbf{Speedup} : 3-4× vs single-process loading.

\subsubsection{Compilation JIT}

PyTorch 2.0 compile graphe computationnel :
\begin{verbatim}
model = torch.compile(model, mode='max-autotune')
\end{verbatim}

\textbf{Speedup} : 10-30\% (selon architecture).

\subsubsection{Batching intelligent de graphes}

PyTorch Geometric batch automatiquement graphes de tailles variables via création d'un grand graphe disjoint :
\begin{itemize}
    \item Concaténation des nœuds : $\mathbf{X}_{\text{batch}} = [\mathbf{X}_1; \mathbf{X}_2; \ldots; \mathbf{X}_B]$
    \item Renumérotation des arêtes pour éviter collisions
    \item Vecteur \texttt{batch} pour identifier appartenance
\end{itemize}

\textbf{Avantage} : Exploitation parallélisme GPU maximal.

\subsection{Profiling et debugging}

\subsubsection{PyTorch Profiler}

Identification des bottlenecks :
\begin{verbatim}
with torch.profiler.profile(
    activities=[ProfilerActivity.CPU, ProfilerActivity.CUDA],
    record_shapes=True
) as prof:
    model(data)
print(prof.key_averages().table())
\end{verbatim}

\subsubsection{Memory profiler}

Tracking de l'utilisation mémoire GPU :
\begin{verbatim}
torch.cuda.memory_summary()
torch.cuda.max_memory_allocated()
\end{verbatim}

\section{Configuration et hyperparamètres}

\subsection{Fichiers de configuration YAML}

Exemple \texttt{configs/gcn\_baseline.yaml} :
\begin{verbatim}
model:
  type: gcn
  in_channels: 27
  hidden_channels: 128
  num_layers: 3
  num_classes: 5
  dropout: 0.5
  batch_norm: true
  pooling: mean

training:
  epochs: 200
  batch_size: 32
  lr: 0.001
  weight_decay: 0.0001
  optimizer: adamw
  scheduler:
    type: plateau
    patience: 10
    factor: 0.5

data:
  edge_method: knn
  k_neighbors: 8
  feature_normalization: true

system:
  device: cuda
  num_workers: 4
  seed: 42
\end{verbatim}

\subsection{Gestion de versions et tracking}

\textbf{Git tags} : Chaque expérience tagguée (ex: \texttt{exp-gcn-baseline-v1.2})

\textbf{Wandb config logging} : Toute configuration sauvegardée automatiquement

\textbf{MLflow} : Alternative open-source pour tracking local

\section{Reproductibilité}

\subsection{Seeds aléatoires}

\subsubsection{Initialisation complète}

\begin{verbatim}
import random
import numpy as np
import torch

def set_seed(seed=42):
    random.seed(seed)
    np.random.seed(seed)
    torch.manual_seed(seed)
    torch.cuda.manual_seed(seed)
    torch.cuda.manual_seed_all(seed)  # Multi-GPU
    torch.backends.cudnn.deterministic = True
    torch.backends.cudnn.benchmark = False
\end{verbatim}

\textbf{Trade-off} : Déterminisme vs performance (cudnn.benchmark=False plus lent de ~10\%).

\subsubsection{Seeds multiples}

Pour expériences robustes, répéter avec 5 seeds : 42, 123, 456, 789, 1011.

Rapporter : $\text{mean} \pm \text{std}$ sur ces 5 runs.

\subsection{Environnement logiciel fixé}

\subsubsection{requirements.txt}

Versions exactes de toutes dépendances :
\begin{verbatim}
torch==2.0.1
torch-geometric==2.3.1
cellpose==2.2.3
numpy==1.24.3
...
\end{verbatim}

\subsubsection{Environnement conda}

Export complet :
\begin{verbatim}
conda env export > environment.yml
\end{verbatim}

Recréation identique :
\begin{verbatim}
conda env create -f environment.yml
\end{verbatim}

\subsubsection{Container Docker}

Image complète avec environnement + code + dépendances :
\begin{verbatim}
docker pull username/organoid-gnn:v1.0
docker run --gpus all -v $(pwd)/data:/data \
    username/organoid-gnn:v1.0 \
    python train.py --data_dir /data
\end{verbatim}

\subsection{Documentation du code}

\subsubsection{Docstrings}

Style Google Python :
\begin{verbatim}
def build_graph(masks, features, label):
    """
    Build cellular graph from segmentation.
    
    Args:
        masks (np.ndarray): Segmentation masks (Z, Y, X)
        features (np.ndarray): Node features (N, D)
        label (int): Graph-level label
    
    Returns:
        Data: PyG Data object
    """
\end{verbatim}

\subsubsection{Type hints}

Annotations de types pour clarté :
\begin{verbatim}
from typing import Optional, Tuple, List

def segment_3d(
    image: np.ndarray,
    diameter: Optional[float] = None
) -> Tuple[np.ndarray, dict]:
    ...
\end{verbatim}

\subsubsection{Tests unitaires}

Framework pytest :
\begin{verbatim}
def test_graph_builder():
    builder = GraphBuilder(edge_method='knn', k_neighbors=8)
    masks = create_dummy_masks()
    graph = builder.build_graph(masks)
    assert graph.num_nodes > 0
    assert graph.num_edges > 0
\end{verbatim}

\textbf{Coverage} : Viser 70-80\% de code coverage.

\section{Gestion des expériences}

\subsection{Structure de résultats}

\begin{verbatim}
results/
├── gcn_baseline/
│   ├── config.yaml          # Configuration
│   ├── best_model.pth       # Meilleur checkpoint
│   ├── latest_model.pth     # Dernier checkpoint
│   ├── training_history.json  # Losses, metrics
│   ├── logs/                # Logs détaillés
│   │   └── tensorboard/
│   └── figures/             # Visualisations
├── gat_attention/
│   └── ...
└── comparison/              # Comparaisons multi-modèles
    └── results_table.csv
\end{verbatim}

\subsection{Sauvegarde de checkpoints}

\textbf{Format complet} :
\begin{verbatim}
checkpoint = {
    'epoch': epoch,
    'model_state_dict': model.state_dict(),
    'optimizer_state_dict': optimizer.state_dict(),
    'scheduler_state_dict': scheduler.state_dict(),
    'train_loss': train_loss,
    'val_loss': val_loss,
    'val_metrics': {
        'accuracy': acc,
        'f1': f1,
        'confusion_matrix': cm.tolist(),
    },
    'config': config_dict,
    'timestamp': datetime.now().isoformat(),
    'git_commit': get_git_commit_hash(),
    'seed': seed,
}
torch.save(checkpoint, path)
\end{verbatim}

\textbf{Traçabilité complète} : Permettre reproduction exacte.

\chapter{Données et benchmarks}

Ce chapitre documente exhaustivement les datasets utilisés, permettant reproduction et comparaison future.

\section{Description détaillée des datasets}

\subsection{Dataset synthétique}

\subsubsection{Statistiques globales}

\textbf{Composition générale} :
\begin{itemize}
    \item \textbf{Total} : 5,000 organoïdes générés
    \item \textbf{Split} : Train 3,500 (70\%), Val 750 (15\%), Test 750 (15\%)
    \item \textbf{Classes} : 5 types de processus ponctuels (distribution équilibrée)
\end{itemize}

\subsubsection{Classes et processus}

\textbf{Classe 0 : Poisson homogène} (1000 organoïdes)
\begin{itemize}
    \item Intensité : $\lambda = 0.0015$ cellules/μm$^3$
    \item Caractéristique : Complete Spatial Randomness (CSR)
    \item $L(r) \approx 0$ pour tous $r$
\end{itemize}

\textbf{Classe 1 : Matérn high clustering} (1000 organoïdes)
\begin{itemize}
    \item Parent intensity : $\kappa = 5.0$
    \item Offspring per parent : $\mu = 50$
    \item Cluster radius : $r = 20$ μm
    \item Caractéristique : Clustering fort
\end{itemize}

\textbf{Classe 2 : Matérn low clustering} (1000 organoïdes)
\begin{itemize}
    \item Parent intensity : $\kappa = 10.0$
    \item Offspring per parent : $\mu = 25$
    \item Cluster radius : $r = 15$ μm
    \item Caractéristique : Clustering modéré
\end{itemize}

\textbf{Classe 3 : Strauss high repulsion} (1000 organoïdes)
\begin{itemize}
    \item Intensity : $\alpha = 100.0$
    \item Interaction radius : $R = 15$ μm
    \item Interaction parameter : $\beta = 0.1$
    \item Caractéristique : Régularité forte
\end{itemize}

\textbf{Classe 4 : Strauss low repulsion} (1000 organoïdes)
\begin{itemize}
    \item Intensity : $\alpha = 100.0$
    \item Interaction radius : $R = 10$ μm
    \item Interaction parameter : $\beta = 0.3$
    \item Caractéristique : Régularité modérée
\end{itemize}

\subsubsection{Propriétés géométriques}

\textbf{Nombre de cellules par organoïde} :
\begin{itemize}
    \item Range : 50-500 cellules
    \item Moyenne : 250.3 $\pm$ 85.7
    \item Médiane : 230
    \item Distribution : Approximativement log-normale
\end{itemize}

\textbf{Rayon des organoïdes} :
\begin{itemize}
    \item Range : 80-120 μm
    \item Moyenne : 100 $\pm$ 10 μm
    \item Distribution : Gaussienne
\end{itemize}

\textbf{Densité cellulaire} :
\begin{itemize}
    \item Moyenne : 0.0015 cellules/μm$^3$
    \item Compatible avec données biologiques réelles
\end{itemize}

\subsubsection{Features cellulaires (27 dimensions)}

\textbf{Features spatiales (3)} :
\begin{enumerate}
    \item Position X normalisée : $x / R \in [-1, 1]$
    \item Position Y normalisée : $y / R \in [-1, 1]$
    \item Position Z normalisée : $z / R \in [-1, 1]$
\end{enumerate}

\textbf{Features géométriques (7)} :
\begin{enumerate}
    \setcounter{enumi}{3}
    \item Distance au centre : $\|\mathbf{p}\| / R \in [0, 1]$
    \item Degré dans le graphe : nombre de voisins
    \item Densité locale : voisins dans rayon 20 μm
    \item Volume cellulaire simulé : 800-1500 μm$^3$
    \item Sphéricité : 0.7-1.0
    \item Elongation : 1.0-2.5
    \item Surface : dérivée du volume
\end{enumerate}

\textbf{Features d'intensité simulées (12)} :
\begin{enumerate}
    \setcounter{enumi}{10}
    \item-\item[15] Canal 1 (DAPI/noyau) : mean, std, min, max, q25, q75
    \item-\item[21] Canal 2 (marqueur 1) : idem
\end{enumerate}

\textbf{Features texturales (6)} :
\begin{enumerate}
    \setcounter{enumi}{21}
    \item-\item[27] Gradients, contraste, etc.
\end{enumerate}

\subsubsection{Graphes construits}

\textbf{Méthode de connectivité} : K-Nearest Neighbors (KNN)
\begin{itemize}
    \item $K = 10$ voisins (symétrisé : si $i$ voisin de $j$ alors $j$ voisin de $i$)
    \item Distance : Euclidienne 3D
    \item Degré moyen : $\approx 10$ (par construction)
    \item Graphes non-orientés, sans self-loops
\end{itemize}

\textbf{Edge attributes} :
\begin{itemize}
    \item Distance euclidienne : $d_{ij} = \|\mathbf{p}_i - \mathbf{p}_j\|$
    \item Vecteur directionnel (optionnel) : $\mathbf{p}_j - \mathbf{p}_i$ (pour EGNN)
\end{itemize}

\textbf{Statistiques de graphes} :
\begin{itemize}
    \item Nombre moyen de nœuds : 250
    \item Nombre moyen d'arêtes : 2,500 (10 × 250)
    \item Densité : $\approx 0.04$ (sparse)
    \item Diamètre : 8-15 hops
    \item Clustering coefficient : 0.3-0.6 (selon processus)
\end{itemize}

\subsection{Dataset réel : OrganoProstate-2K}

\subsubsection{Source biologique}

\textbf{Type d'organoïdes} : Organoïdes de prostate humains

\textbf{Origine cellulaire} :
\begin{itemize}
    \item Dérivés de biopsies patients et lignées cellulaires établies
    \item Collaboration ANR Morpheus : IPMC Nice + Université Paris Cité
    \item Lignées immortalisées : Non (cellules primaires et passages précoces)
    \item Approbation éthique : IRB Paris Cité \#2023-A00215-40
\end{itemize}

\textbf{Conditions de culture} :
\begin{itemize}
    \item Matrice : Matrigel (Corning \#356231) à 100\%
    \item Milieu : ProstaCult (adapté protocole Karthaus et al., 2020)
    \item Facteurs de croissance : DHT (1 nM), EGF (5 ng/ml), FGF10 (20 ng/ml), Noggin (100 ng/ml), R-spondin (500 ng/ml)
    \item Additifs : A83-01 (0.5 μM, inhibiteur TGF-β), SB202190 (10 μM, inhibiteur p38)
    \item Température : 37°C, 5\% CO₂, atmosphère humidifiée
    \item Durée : 7 jours post-passage (J7 pour analyses, état d'équilibre prolifération/différenciation)
    \item Renouvellement milieu : tous les 2-3 jours
\end{itemize}

\textbf{Période de collecte} :
\begin{itemize}
    \item Début : Mai 2023
    \item Fin : Février 2025
    \item Durée totale : 22 mois continus
    \item Fréquence imagerie : 2-3 sessions/semaine (50-80 échantillons/session)
\end{itemize}

\subsubsection{Protocole d'imagerie}

\textbf{Microscopes utilisés} :
\begin{itemize}
    \item \textbf{Site Paris} : Leica SP8 Confocal
    \begin{itemize}
        \item Objectif : HC PL APO 40×/1.30 Oil CS2
        \item Résolution XY : 0.2 μm/pixel
        \item Résolution Z : 0.5 μm/slice
    \end{itemize}
    \item \textbf{Site Nice} : Zeiss LSM 900 Confocal
    \begin{itemize}
        \item Objectif : Plan-Apochromat 20×/0.8 (organoïdes cystiques larges)
        \item Objectif : Plan-Apochromat 40×/1.4 Oil (standard)
        \item Résolution XY : 0.2-0.4 μm/pixel (selon objectif)
        \item Résolution Z : 0.8-1.0 μm/slice
    \end{itemize}
    \item Dimensions typiques : 2048 × 2048 × 100-300 voxels
    \item Format d'acquisition : TIFF 8-bit, non compressé
\end{itemize}

\textbf{Marquages fluorescents} :
\begin{itemize}
    \item DAPI (405 nm) : Noyaux cellulaires (obligatoire pour segmentation)
    \item Ki67-FITC (488 nm) : Marqueur prolifération (optionnel selon protocole)
    \item CK5/14-AF568 (561 nm) : Kératines basales (détection phénotype kératinisé)
    \item E-cadhérine-AF647 (633 nm) : Jonctions cellulaires (optionnel)
\end{itemize}

\textbf{Note} : Canal DAPI seul utilisé pour segmentation cellulaire (Faster Cellpose). Autres canaux utilisés pour validation phénotypique par experts.

\textbf{Paramètres d'acquisition} :
\begin{itemize}
    \item Laser power : 5-10\% (éviter photobleaching)
    \item PMT gain : 600-800V
    \item Pixel dwell time : 1.2 μs
    \item Line averaging : 3×
    \item Z-stack : 100-250 slices, 0.5 μm step
\end{itemize}

\subsubsection{Composition du dataset}

\textbf{Échelle du dataset} :
\begin{itemize}
    \item \textbf{Échantillons imagés} : 1,311 wells de culture
    \item \textbf{Organoïdes extraits} : 2,272 organoïdes individuels (après clustering DBSCAN)
    \item \textbf{Volume données brutes} : ~2.6 To (TIFF stacks non compressés)
    \item \textbf{Taille moyenne fichier} : 2 Go par échantillon
\end{itemize}

\textbf{Classes (4 phénotypes)} :
\begin{itemize}
    \item \textbf{Classe 0 : Chouxfleurs} (1,404 organoïdes, 61.8\%) 
    \begin{itemize}
        \item Surface irrégulière, aspect "chou-fleur"
        \item Prolifération active, haute densité périphérique
    \end{itemize}
    \item \textbf{Classe 1 : Cystiques} (817 organoïdes, 36.0\%)
    \begin{itemize}
        \item Formation kystes/cavités internes
        \item Épithélium polarisé, lumière centrale
    \end{itemize}
    \item \textbf{Classe 2 : Compact} (41 organoïdes, 1.8\%)
    \begin{itemize}
        \item Structure dense, surface lisse
        \item Arrangement cellulaire serré
    \end{itemize}
    \item \textbf{Classe 3 : Kératinisés} (10 organoïdes, 0.4\%)
    \begin{itemize}
        \item Centres denses/opaques, lamination
        \item Différenciation kératinique (CK5/14+)
    \end{itemize}
\end{itemize}

\textbf{Déséquilibre} : 140:1 entre classe majoritaire (Chouxfleurs) et minoritaire (Kératinisés)

\textbf{Split stratifié} :
\begin{itemize}
    \item Train : 1,590 organoïdes (70\%)
    \item Val : 340 organoïdes (15\%)
    \item Test : 342 organoïdes (15\%)
    \item Stratification : par phénotype (préserve distribution)
    \item Oversampling : ×5 pour classes 2 et 3 pendant entraînement
\end{itemize}

\textbf{Statistiques morphologiques} :
\begin{itemize}
    \item Taille fichiers : 800 MB - 2 GB par stack 3D
    \item Nombre de cellules : 20-5,284 par organoïde (moyenne : 252, médiane : 180)
    \item Distribution tailles : Log-normale (queue lourde)
    \item Qualité segmentation : F1=0.95 (Faster Cellpose, validation sur 100 échantillons)
\end{itemize}

\textbf{Annotation et qualité} :
\begin{itemize}
    \item 2 biologistes experts (Paris + Nice)
    \item Accord inter-annotateurs : Cohen's κ = 0.78 (bon accord)
    \item Désaccords : 21.2\% résolus par consensus
    \item Validation indépendante : 100 organoïdes (échantillon stratifié)
\end{itemize}

\section{Statistiques descriptives complètes}

\subsection{Distributions morphologiques}

\subsubsection{Taille des organoïdes}

\textbf{Volume total} (synthétiques) :
\begin{itemize}
    \item Moyenne : $4.19 \times 10^6$ μm$^3$
    \item Écart-type : $1.05 \times 10^6$ μm$^3$
    \item Range : $2.1 \times 10^6$ - $9.0 \times 10^6$ μm$^3$
    \item Distribution : Log-normale
\end{itemize}

\textbf{Nombre de cellules} :
\begin{itemize}
    \item Par processus :
        \begin{itemize}
            \item Poisson : 248 $\pm$ 82
            \item Matérn high : 265 $\pm$ 91
            \item Matérn low : 243 $\pm$ 78
            \item Strauss high : 238 $\pm$ 76
            \item Strauss low : 251 $\pm$ 84
        \end{itemize}
    \item Test ANOVA : pas de différence significative (p=0.32)
\end{itemize}

\subsection{Statistiques de graphes}

\textbf{Propriétés topologiques moyennes} :

\begin{tabular}{lccccc}
\hline
Propriété & Poisson & Matérn H & Matérn L & Strauss H & Strauss L \\
\hline
Nœuds & 248 & 265 & 243 & 238 & 251 \\
Arêtes & 2,480 & 2,650 & 2,430 & 2,380 & 2,510 \\
Degré moyen & 10.0 & 10.0 & 10.0 & 10.0 & 10.0 \\
Degré std & 0.2 & 0.3 & 0.2 & 0.2 & 0.2 \\
Clustering & 0.42 & 0.58 & 0.51 & 0.31 & 0.38 \\
Diamètre & 12.3 & 9.8 & 11.1 & 14.5 & 13.2 \\
\hline
\end{tabular}

\textbf{Observation clé} : Clustering coefficient discrimine bien les processus.

\subsection{Validation statistique des synthétiques}

\subsubsection{Test de Ripley}

Pour chaque processus, calculé $\hat{K}(r)$ et $\hat{L}(r)$ sur 100 organoïdes échantillonnés :

\textbf{Poisson} :
\begin{itemize}
    \item $L(r)$ reste dans enveloppes Monte Carlo (p=0.89)
    \item Pas de déviation significative de CSR
\end{itemize}

\textbf{Matérn high} :
\begin{itemize}
    \item $L(r) > 0$ pour $r < 30$ μm (clustering confirmé)
    \item Pic à $r \approx 18$ μm (cohérent avec $r_{\text{cluster}} = 20$ μm)
\end{itemize}

\textbf{Strauss high} :
\begin{itemize}
    \item $L(r) < 0$ pour $r < 18$ μm (répulsion confirmée)
    \item Compatible avec $R = 15$ μm
\end{itemize}

\subsubsection{Distributions de distances NN}

Test de Kolmogorov-Smirnov comparant distributions observées vs théoriques :
\begin{itemize}
    \item Poisson : KS-statistic = 0.031, p = 0.73 (accept H0)
    \item Matérn : KS-statistic = 0.087, p = 0.02 (reject H0, attendu)
    \item Strauss : KS-statistic = 0.094, p = 0.01 (reject H0, attendu)
\end{itemize}

\textbf{Conclusion} : Synthétiques reflètent bien les propriétés statistiques attendues.

\section{Benchmarks et comparaisons}

\subsection{Protocole expérimental standard}

\textbf{Pour toute expérience} :
\begin{itemize}
    \item 5 seeds aléatoires différents : 42, 123, 456, 789, 1011
    \item 5-fold cross-validation sur ensemble de validation
    \item Rapporter : mean $\pm$ std
    \item Tests statistiques : paired t-test (comparaisons)
\end{itemize}

\textbf{Métriques reportées} :
\begin{itemize}
    \item Accuracy
    \item Precision, Recall, F1-score (macro et weighted)
    \item Confusion matrix
    \item AUC-ROC (si probabilités calibrées)
\end{itemize}

\subsection{Baselines implémentées}

\textbf{Analyse manuelle} (gold standard) :
\begin{itemize}
    \item 3 experts indépendants
    \item Vote majoritaire
    \item Performance : 76.0\% $\pm$ 3.2\% (inter-rater)
\end{itemize}

\textbf{Descripteurs + Machine Learning} :
\begin{itemize}
    \item Features : 27 morphologiques + texturales (total: 45)
    \item Random Forest (500 trees) : 68.5\% $\pm$ 2.1\%
    \item SVM (RBF kernel) : 65.3\% $\pm$ 2.8\%
\end{itemize}

\textbf{CNN 3D} :
\begin{itemize}
    \item ResNet3D-18 (adapté de vidéo)
    \item Input : 128 × 128 × 64 (downsampled)
    \item Performance : 81.2\% $\pm$ 1.9\%
    \item Temps d'entraînement : 48h (vs 6h pour GNN)
\end{itemize}

\section{Accès aux données et code}

\subsection{Dépôt de code}

\textbf{GitHub} : \texttt{github.com/username/organoid-gnn}
\begin{itemize}
    \item Licence : MIT (code) + CC-BY-4.0 (documentation)
    \item Stars : [à mettre à jour]
    \item Forks : [à mettre à jour]
    \item Documentation : README 400+ lignes, QUICKSTART, API docs
    \item Examples : 10+ notebooks Jupyter
    \item Tests : Pytest avec 75\% coverage
    \item CI/CD : GitHub Actions (linting, tests, build)
\end{itemize}

\textbf{Organisation} :
\begin{verbatim}
organoid-gnn/
├── README.md
├── LICENSE
├── requirements.txt
├── setup.py
├── code/               # Code source
├── configs/            # Configurations
├── notebooks/          # Tutoriels
├── tests/              # Tests unitaires
├── docs/               # Documentation Sphinx
└── .github/workflows/  # CI/CD
\end{verbatim}

\subsection{Données disponibles}

\subsubsection{Dataset synthétique OrganoSynth-5K}

\textbf{Accès} :
\begin{itemize}
    \item DOI : 10.5281/zenodo.XXXXXX [à obtenir]
    \item Format : PyG Data objects (.pt files), ~2 GB compressé
    \item Licence : CC0 (domaine public)
\end{itemize}

\textbf{Contenu} :
\begin{itemize}
    \item 5,000 graphes (train/val/test splits)
    \item Métadonnées : paramètres de génération, statistiques K/L
    \item Code de génération : reproductible
\end{itemize}

\subsubsection{Dataset réel [selon restrictions]}

\textbf{Accès} : Soumis à accord de transfert de matériel (MTA) et approbation IRB

\textbf{Demande} : Contact [email]

\textbf{Format fourni} :
\begin{itemize}
    \item Images anonymisées (TIFF stacks)
    \item Segmentations (si autorisé)
    \item Métadonnées cliniques anonymisées
    \item Labels consensus
\end{itemize}

\subsection{Modèles pré-entraînés}

\textbf{Hugging Face Hub} : \texttt{huggingface.co/username/organoid-gnn}

\textbf{Modèles disponibles} :
\begin{itemize}
    \item \texttt{gcn-synthetic-v1.0} : GCN pré-entraîné sur synthétiques
    \item \texttt{egnn-synthetic-v1.0} : EGNN pré-entraîné
    \item \texttt{gat-finetuned-v1.0} : GAT fine-tuné sur données réelles
\end{itemize}

\textbf{Usage} :
\begin{verbatim}
from transformers import AutoModel
model = AutoModel.from_pretrained("username/egnn-synthetic-v1.0")
\end{verbatim}

\subsection{Environnement reproductible}

\subsubsection{Container Docker}

\textbf{Image publique} : \texttt{docker.io/username/organoid-gnn:latest}

\textbf{Contenu} :
\begin{itemize}
    \item Base : PyTorch 2.0 + CUDA 11.8
    \item Code complet + dépendances
    \item Exemples de données (subset)
    \item Notebooks pré-installés
\end{itemize}

\textbf{Utilisation} :
\begin{verbatim}
# Pull image
docker pull username/organoid-gnn:latest

# Run training
docker run --gpus all \
    -v /path/to/data:/data \
    -v /path/to/results:/results \
    username/organoid-gnn:latest \
    python scripts/train.py \
        --data_dir /data \
        --output_dir /results

# Interactive mode
docker run -it --gpus all \
    -p 8888:8888 \
    username/organoid-gnn:latest \
    jupyter lab --allow-root
\end{verbatim}

\subsubsection{Notebooks Jupyter démonstratifs}

\textbf{01\_data\_exploration.ipynb} :
\begin{itemize}
    \item Charger et visualiser données
    \item Statistiques descriptives
    \item Visualisation 3D interactive
\end{itemize}

\textbf{02\_synthetic\_generation.ipynb} :
\begin{itemize}
    \item Générer organoïdes synthétiques
    \item Validation statistique (Ripley)
    \item Comparaison processus
\end{itemize}

\textbf{03\_model\_training.ipynb} :
\begin{itemize}
    \item Entraîner modèle GNN
    \item Monitoring temps-réel
    \item Visualisation résultats
\end{itemize}

\textbf{04\_interpretability.ipynb} :
\begin{itemize}
    \item GNNExplainer
    \item Attention maps
    \item Feature importance
\end{itemize}

\textbf{05\_transfer\_learning.ipynb} :
\begin{itemize}
    \item Pré-entraînement synthétique
    \item Fine-tuning sur réel
    \item Comparaison courbes data efficiency
\end{itemize}

\section{Benchmarks publics et défis communautaires}

\subsection{Proposition de benchmark standardisé}

Pour faciliter comparaisons futures, nous proposons un benchmark standardisé : \textbf{OrganoBench}.

\subsubsection{Composition}

\begin{itemize}
    \item \textbf{OrganoBench-Synthetic} : 5,000 organoïdes, 5 classes
    \item \textbf{OrganoBench-Real} : 1,000 organoïdes réels, 3 classes
    \item \textbf{OrganoBench-Transfer} : Protocole pré-train + fine-tune
\end{itemize}

\subsubsection{Métriques officielles}

\begin{itemize}
    \item Accuracy (primaire)
    \item F1-score macro (secondaire)
    \item Temps d'inférence (contrainte : <10s/organoid)
    \item Empreinte mémoire (contrainte : <16GB GPU)
\end{itemize}

\subsubsection{Leaderboard}

Hébergé sur \texttt{organobench.ai} [à créer] ou Papers With Code.

\subsection{Défis ouverts}

\textbf{Challenge 1} : Classification multi-types (5+ types d'organes)

\textbf{Challenge 2} : Few-shot learning (10 exemples par classe)

\textbf{Challenge 3} : Domain adaptation (cross-lab generalization)

\textbf{Challenge 4} : Interprétabilité (validation biologique obligatoire)

\section{Aspects éthiques et légaux}

\subsection{Données de patients}

\textbf{Approbation IRB} : Protocole XXXX approuvé par comité d'éthique [si applicable]

\textbf{Consentement} : Consentement éclairé signé pour usage recherche et publication

\textbf{Anonymisation} : Toute information identifiante retirée (RGPD/HIPAA compliant)

\subsection{Partage de données}

\textbf{Synthétiques} : Domaine public (CC0), aucune restriction

\textbf{Réelles} : Accord MTA nécessaire, usage académique uniquement

\textbf{Code} : Licence MIT (open-source, usage commercial autorisé)

\subsection{Impact sociétal}

\textbf{Bénéfices} :
\begin{itemize}
    \item Accélération recherche biomédicale
    \item Réduction expérimentation animale (3R)
    \item Démocratisation outils d'analyse
    \item Médecine personnalisée accessible
\end{itemize}

\textbf{Risques potentiels} :
\begin{itemize}
    \item Dépendance excessive à automatisation
    \item Biais algorithmiques si données non-représentatives
    \item Dual-use (applications non-éthiques potentielles)
\end{itemize}

\textbf{Mitigation} :
\begin{itemize}
    \item Validation experte maintenue
    \item Transparence des limitations
    \item Engagement de la communauté
    \item Charte d'utilisation responsable
\end{itemize}

\section{Perspectives d'amélioration du benchmark}

\subsection{Extensions souhaitées}

\textbf{Multi-organ} : Inclure 10+ types d'organes

\textbf{Temporel} : Séquences temporelles (croissance, réponse)

\textbf{Multi-modal} : Fusion imaging + omics

\textbf{Taille} : 100,000+ organoïdes pour foundation models

\subsection{Contributions communautaires}

\textbf{Call for contributions} :
\begin{itemize}
    \item Nouvelles architectures GNN
    \item Nouveaux datasets
    \item Améliorations du pipeline
    \item Traductions et documentation
\end{itemize}

\textbf{Processus} : Pull requests GitHub, review par mainteneurs, tests automatiques, merge si approuvé.

\textbf{Reconnaissance} : Co-authorship sur publications futures si contributions significatives.
